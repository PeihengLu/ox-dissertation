% First parameter can be changed eg to "Glossary" or something.
% Second parameter is the max length of bold terms.
\begin{mclistof}{List of Abbreviations}{3.2cm}

    \item[ssDNA/dsDNA] Single Stranded DNA/Double Stranded DNA
    
    \item[cDNA] Complementary DNA (of a RNA sequence)
    
    \item[nt] Nucleotide, the basic building block of DNA and RNA, often used as a unit of length in DNA/RNA sequences.
    
    \item[bp] Base pair, also a unit of length in sequences, but more specifically refers to the length of a double stranded molecules.   

    \item[RT] Reverse Transcriptase. An enzyme that synthesizes DNA from RNA templates.

    \item[PBS] Prime Binding Site. Used to create a short sequence of annealed DNA-RNA pairs and prime the reverse transcription process. 

    \item[RTT] Reverse Transcription Template. A RNA sequence that is reverse transcribed into DNA by the RT enzyme.

    \item[LHA/RHA] The Left and Right Homology Arms. These are part of the RTT sequences that are homologous to the target DNA sequence. The left arm is adjacent to the PBS and the right arm is adjacent to the crRNA scaffold. RHA is also referred to as RTT overhang in some literature.
    
    \item[CNN] Convolutional Neural Network. A type of neural network that is often used for image recognition tasks.
    
    \item[RNN] Recurrent Neural Network. A type of neural network that is often used for sequence data.
    
    \item[GRU] Gated Recurrent Unit. A type of RNN that is designed to capture long-term dependencies in sequences.

\end{mclistof} 