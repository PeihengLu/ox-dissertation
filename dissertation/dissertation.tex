% This one will format for two-sided binding (ie left and right pages have mirror margins; blank pages inserted where needed):
\documentclass[a4paper,12pt]{ociamthesis}
% This one will format for one-sided binding (ie left margin > right margin; no extra blank pages):
%\documentclass[a4paper]{ociamthesis}
% This one will format for PDF output (ie equal margins, no extra blank pages):
%\documentclass[a4paper,nobind]{ociamthesis} 
\pdfminorversion=7



%%%%% SELECT YOUR DRAFT OPTIONS
% Three options going on here; use in any combination.  But remember to turn the first two off before
% generating a PDF to send to the printer!

% This adds a "DRAFT" footer to every normal page.  (The first page of each chapter is not a "normal" page.)
% \fancyfoot[C]{\emph{DRAFT Printed on \today}}  

% This highlights (in blue) corrections marked with (for words) \mccorrect{blah} or (for whole
% paragraphs) \begin{mccorrection} . . . \end{mccorrection}.  This can be useful for sending a PDF of
% your corrected thesis to your examiners for review.  Turn it off, and the blue disappears.
\correctionstrue

\usepackage{subfigure}
\usepackage{listings}
\usepackage{amssymb}


%%%%% BIBLIOGRAPHY SETUP
% The science-type option: numerical in-text citation with references in order of appearance.
\usepackage[style=numeric-comp, sorting=none, backend=biber, doi=false, isbn=false, language=british]{biblatex}
\newcommand*{\bibtitle}{References}

% This makes the bibliography left-aligned (not 'justified') and slightly smaller font.
\renewcommand*{\bibfont}{\raggedright\small}

% Change this to the name of your .bib file (usually exported from a citation manager like Zotero or EndNote).
\addbibresource{references.bib}
% graphics path
\graphicspath{ {./figures/} }


% Uncomment this if you want equation numbers per section (2.3.12), instead of per chapter (2.18):
%\numberwithin{equation}{subsection}



%%%%% THESIS / TITLE PAGE INFORMATION
% Everybody needs to complete the following:
\title{Prediction Prime Editing Outcome with Machine Learning}
\author{Peiheng Lu}
\college{Wolfson College}

% Master's candidates who require the alternate title page (with candidate number and word count)
% must also un-comment and complete the following three lines:
\masterssubmissiontrue
\candidateno{1592800}
\wordcount{10,300}

% Uncomment the following line if your degree also includes exams (eg most masters):
%\renewcommand{\submittedtext}{Submitted in partial completion of the}
% Your full degree name.  (But remember that DPhils aren't "in" anything.  They're just DPhils.)
\degree{MSc Advanced Computer Science}
% Term and year of submission, or date if your board requires (eg most masters)
\degreedate{Michaelmas 2024}


%%%%% YOUR OWN PERSONAL MACROS
% This is a good place to dump your own LaTeX macros as they come up.

% To make text superscripts shortcuts
\renewcommand{\th}{\textsuperscript{th}} % ex: I won 4\th place
\newcommand{\nd}{\textsuperscript{nd}}
\renewcommand{\st}{\textsuperscript{st}}
\newcommand{\rd}{\textsuperscript{rd}}

%%%%% THE ACTUAL DOCUMENT STARTS HERE
\begin{document}

%%%%% CHOOSE YOUR LINE SPACING HERE
% This is the official option.  Use it for your submission copy and library copy:
\setlength{\textbaselineskip}{22pt plus2pt}
% This is closer spacing (about 1.5-spaced) that you might prefer for your personal copies:
%\setlength{\textbaselineskip}{18pt plus2pt minus1pt}

% You can set the spacing here for the roman-numbered pages (acknowledgements, table of contents, etc.)
\setlength{\frontmatterbaselineskip}{18pt plus2pt minus1pt}

% Leave this line alone; it gets things started for the real document.
\setlength{\baselineskip}{\textbaselineskip}

%%%%% CHOOSE YOUR SECTION NUMBERING DEPTH HERE
% You have two choices.  First, how far down are sections numbered?  (Below that, they're named but
% don't get numbers.)  Second, what level of section appears in the table of contents?  These don't have
% to match: you can have numbered sections that don't show up in the ToC, or unnumbered sections that
% do.  Throughout, 0 = chapter; 1 = section; 2 = subsection; 3 = subsubsection, 4 = paragraph...

% The level that gets a number:
\setcounter{secnumdepth}{2}
% The level that shows up in the ToC:
\setcounter{tocdepth}{2}


%%%%% ABSTRACT SEPARATE
% This is used to create the separate, one-page abstract that you are required to hand into the Exam
% Schools.  You can comment it out to generate a PDF for printing or whatnot.
% \begin{abstractseparate}

% \end{abstractseparate}


% JEM: Pages are roman numbered from here, though page numbers are invisible until ToC.  This is in
% keeping with most typesetting conventions.
\begin{romanpages}

% JEM: By default, this template uses the traditional Oxford "Belt Crest". Un-comment the following
% line to use the newer, "Blue Square" logo:
\renewcommand{\crest}{{\includegraphics[width=4.2cm, height=4.2cm]{figures/newlogo.pdf}}}

% Title page is created here
\maketitle

%%%%% DEDICATION -- If you'd like one, un-comment the following.
%\begin{dedication}
%This thesis is dedicated to\\
%someone\\
%for some special reason\\
%\end{dedication}

%%%%% ACKNOWLEDGEMENTS -- Nothing to do here except comment out if you don't want it.
\begin{acknowledgements}
    
\end{acknowledgements}

%%%%% ABSTRACT -- Nothing to do here except comment out if you don't want it.
\begin{abstract}

\end{abstract}

%%%%% MINI TABLES
% This lays the groundwork for per-chapter, mini tables of contents.  Comment the following line
% (and remove \minitoc from the chapter files) if you don't want this.  Un-comment either of the
% next two lines if you want a per-chapter list of figures or tables.
\dominitoc % include a mini table of contents
%\dominilof  % include a mini list of figures
%\dominilot  % include a mini list of tables

% This aligns the bottom of the text of each page.  It generally makes things look better.
\flushbottom

% This is where the whole-document ToC appears:
% This is where the whole-document ToC appears:
\tableofcontents

\listoffigures
\mtcaddchapter
% \mtcaddchapter is needed when adding a non-chapter (but chapter-like) entity to avoid confusing minitoc

% Uncomment to generate a list of tables:
%\listoftables
%	\mtcaddchapter

%%%%% LIST OF ABBREVIATIONS
% This example includes a list of abbreviations.
% First parameter can be changed eg to "Glossary" or something.
% Second parameter is the max length of bold terms.
\begin{mclistof}{List of Abbreviations}{3.2cm}

    \item[ssDNA/dsDNA] Single Stranded DNA/Double Stranded DNA
    
    \item[cDNA] Complementary DNA (of a RNA sequence)
    
    \item[nt] Nucleotide, the basic building block of DNA and RNA, often used as a unit of length in DNA/RNA sequences.
    
    \item[bp] Base pair, also a unit of length in sequences, but more specifically refers to the length of a double stranded molecules.   

    \item[RT] Reverse Transcriptase. An enzyme that synthesizes DNA from RNA templates.

    \item[PBS] Prime Binding Site. Used to create a short sequence of annealed DNA-RNA pairs and prime the reverse transcription process. 

    \item[RTT] Reverse Transcription Template. A RNA sequence that is reverse transcribed into DNA by the RT enzyme.

    \item[LHA/RHA] The Left and Right Homology Arms. These are part of the RTT sequences that are homologous to the target DNA sequence. The left arm is adjacent to the PBS and the right arm is adjacent to the crRNA scaffold. RHA is also referred to as RTT overhang in some literature.
    
    \item[CNN] Convolutional Neural Network. A type of neural network that is often used for image recognition tasks.
    
    \item[RNN] Recurrent Neural Network. A type of neural network that is often used for sequence data.
    
    \item[GRU] Gated Recurrent Unit. A type of RNN that is designed to capture long-term dependencies in sequences.

\end{mclistof} 
\end{romanpages}


\chapter{Introduction}

\minitoc

\section{Background}

The genome of an organism is a complete set of its DNA (Deoxyribonucleic acid), containing all the information needed to build and maintain the organism. The genome is organized into chromosomes, which are long strands of DNA that are further divided into genes, the basic unit of heredity in living organisms. Each gene is a segment of DNA  that contains the instructions for making a group of specific proteins, which make up the structure of the organism and carry out various functions\cite{BrockBiologyMicroorganisms}.

DNA is a double-stranded molecule that encodes the genetic information through the arrangements of different nucleotides consisting of a phosphate group, a hydroxyl group ($-OH$), a sugar group (deoxyribose) and a nitrogen base. The nitrogen bases differentiate the nucleotides into adenine (A), thymine (T), guanine (G) and cytosine (C), each with different chemical properties and is thus a distinct token in the genetic vocabulary. The nucleotides in the same strand are connected by the process of phosphorylation, where phosphate group and the hydroxyl group combine to form a sugar-phosphate backbone. At the same time, the two strands are connected to each other by hydrogen bonds between the complementary nitrogen bases (A-T, G-C), forming the double helix structure of the DNA. The sequence of the bases on the DNA strand determines the genetic information of an organism, and any changes (mutation) to the sequence can result in mutations that can have notable effects on multiple levels of the organism's biology\cite{BrockBiologyMicroorganisms}.

The ability to precisely introduce mutations of multiple types and lengths into specified locations of the genome has profound influence in medical science and the broader field of biology, and has been a long-standing goal of many researchers. It can help to elucidate the function of genes and their regulatory elements, provide clinical interpretation of human gene variants, and to develop new treatments for diseases including genetic disorders and even cancer\cite{petraityteGenomeEditingMedicine2021,dasCRISPRBasedTherapeutics2022,portoBaseEditingAdvances2020}. 

The discovery of the CRISPR (clustered regularly interspaced short palindromic repeats) and CRISPR-associated protein (Cas) system in the early 2000s was a major step towards this goal. The CRISPR system is originally an adaptive immune systems in bacteria and archaea providing acquired immunity against foreign nucleic acids, using RNA (ribonucleic acid) sequences as guide \cite{jiangCRISPRCas9Structures2017}. RNA sequences are single-stranded molecules that carry genetic information from DNA to the ribosome for protein synthesis. They are similar in structure to DNA, but with uracil (U) replacing thymine (T) as one of the nitrogen bases. As a result, they can also form hydrogen bonds with DNA, and carrying the molecules attached to the RNA to the corresponding DNA sequence\cite{BrockBiologyMicroorganisms}.
% This property forms the basis of the CRISPR system's ability to recognize and bond with foreign DNA.

The CRISPR-Cas system works in two steps. In the adaptation step, the system acquires a short sequence of the foreign DNA (spacers) and derive new complimentary guide RNAs from it. Then, in the interference step, the system uses the acquired guide RNAs to recognize and cleave the foreign DNA with the help of attached Cas proteins\cite{garneauCRISPRCasBacterial2010}. 

Due to its ability to recognize and introduce scissions in specific locations of the genome, the CRISPR system and a particular Cas protein, Cas9, have been harnessed as a powerful tool for genome editing (Figure \ref{fig:crispr-hdr}). CRISPR-Cas9 uses a similar two-step process to introduce mutations into the genome. In the first step, the Cas9 protein is guided to the target site by a typically 20bp long single guide RNA (sgRNA) that is complementary to the target site (protospacer) adjacent to the protospacer adjacent motif (PAM) recognizable by Cas9. In the second step, the Cas9 protein introduces a double-strand break (DSB) at the target site, which is then repaired by the cell's endogenous repair machinery. 

For the desired mutation to be installed, a single or double strand DNA (ssDNA/dsDNA) donor can also be provided during the second step\cite{richardsonEnhancingHomologydirectedGenome2016,jasinRepairStrandBreaks2013}. The endogenous homology directed repair (HDR) system can then use the donor DNA as template and install the intended edit\cite{hsuDevelopmentApplicationsCRISPRCas92014}. However, a competing repair pathway, non-homologous end joining (NHEJ), is more prevalent in mammalian cells and often introduces unintended insertions or deletions (indels) at the target loci\cite{changNonhomologousDNAEnd2017}. Although various improvements to the Cas9 proteins and sgRNA have been made over the years, the CRISPR-Cas9 is still mostly used to introduce disruptions into the genome rather than installing precise edits due to its high indel generation rate\cite{kantorCRISPRCas9DNABaseEditing2020,koeppelPredictionPrimeEditing2023}.

% subfigure side by side
\begin{figure}[ht]
    \centering
    \subfigure[CRISPR-Cas9 HDR]{
        \includegraphics[width=0.46\textwidth]{dissertation-crispr-hdr.png}
        \label{fig:crispr-hdr}
    }
    \subfigure[Base Editor]{
        \includegraphics[width=0.46\textwidth]{dissertation-base-editor.png}
        \label{fig:base-editor}
    }
    \caption[Components of HDR and BE]{Components for \textbf{(a)} CRISPR-Cas9 HDR  and \textbf{(b)} Base Editor  systems. The directions of DNA and RNA are denoted by 3' and 5' ends, based on the numbering of carbon atoms in the deoxyribose molecule (sugar group) forming the backbone of the DNA. The 5' end refers to the end where the phosphate  group attaches to the fifth carbon atom of the deoxyribose, while the 3' end refers to the end where the hydroxyl group  attaches to the third carbon atom.}
    \label{fig:crispr-base-editor}
\end{figure}

To avoid the inefficiencies produced by DSB while still leveraging the targeting capabilities of CRISPR system, the nickase versions of Cas9 proteins were developed. Cas9 nickase is a mutated version of the Cas9 protein that only cleaves the binding (RuvC mutation) or opposite (HNH mutation) strand of the guide RNA\cite{dasCRISPRBasedTherapeutics2022}. David Liu and his team developed the base editing system that uses the RuvC mutated Cas9 nickase to introduce point mutations into the genome (Figure \ref{fig:base-editor})\cite{gaudelliProgrammableBaseEditing2017}. The base editing system consists of a fusion protein of the Cas9 nickase and a deaminase enzyme, in addition to a sgRNA that guides the fusion protein to the target site. The deaminase enzyme converts a specific nucleotide within the editing window to another base, and the Cas9 nickase introduces a nick in the non-edited strand to encourage the cell's endogenous repair machinery to use the edited strand as a template and permanently install the edit into the genome.

The base editors have been improved to be able to introduce point mutation within the editing window with relatively high efficiency\cite{portoBaseEditingAdvances2020}. However, they are only designed to introduce single-nucleotide variants into DNA or RNA, and are not capable of inserting and deleting sequences. To address this limitation in functionality and acquire a truly versatile genome editing tool, Liu's lab further developed prime editors\cite{liudavidr.SearchandreplaceGenomeEditing2019}. Prime editors are capable of introducing all 12 possible types of point mutations, as well as insertion and deletion of sequences up to a couple thousand base pairs in length\cite{linHighefficiencyPrimeEditing2021}. 

Prime editors have more complex components than base editors, consisting of a fusion protein of  SpCas9 (Streptococcus pyogenes, HNH mutated) nickase and a MMLV reverse transcriptase, as well as a prime editing guide RNA (pegRNA), illustrated in Figure \ref{fig:pe-pegrna-plasmids}. The SpCas9  nickase is a popular variant of Cas9 protein in recent years, mostly due to its high targetability (recognizing the common NGG PAM sequence, where N can be any nucleotide) and low off-target effects\cite{waltonUnconstrainedGenomeTargeting2020}. The $5'$ end of the pegRNA is very similar to the sgRNA used by base editors and CRISPR-Cas9 HDR, containing a nicking single guide RNA (ngRNA) for targeting protospacer. On the $3'$ end, however, pegRNA has a unique RNA sequence that primes and encodes the desired edit. The $3'$ sequence can be divided into two parts: the primer binding site (PBS) and the reverse transcription template (RTT) consisting of the intended edit surrounded by the left/right homology arms (LHA/RHA, RHA also called RTT overhang). As their names suggest, the PBS and RTT are used to prime the reverse transcription process, while the LHA and RHA are unedited sequences used to facilitate the integration of the edited sequence into the genome. The two ends of the pegRNA are connected by a tracrRNA scaffold sequence, which should have no participation in the prime editing process. 

\section{Process of Prime Editing}

\label{sec:prime-editing-process}

Numerous prime editors has been developed since the first generation (PE2, PE3) was introduced in 2019, including PE4 and PE5 with additional MLH1dn protein to inhibit the adverse mismatch repair pathway, as well as PE2-max and PE4-max with updated SpCas9 and reverse transcriptase\cite{liuPrimeEditingPrecise2023}. Despite their differences, all of the prime editors follow a similar multi-stage process illustrated in \autoref{fig:pe-pegrna-plasmids} and \autoref{fig:prime-editing-process}. 

\begin{figure}
    \centering
    \includegraphics[width=0.85\textwidth]{dissertation-pe-pegrna-plasmids.png}
    \caption[Plasmids for Prime Editing]{The process of introducing prime editors and pegRNA plasmids into the cell. The prime editor plasmid contains the blueprint for the SpCas9 nickase and the MMLV reverse transcriptase, while the pegRNA plasmid contains the cDNA for pegRNA sequence. The sections of the plasmids not responsible for the prime editing process are omitted and replaced by an arrow. The two plasmids are transfected into the cell separately, and the prime editor complex is assembled in the cell's cytoplasm after the expression of PE and transcription of pegRNA.}
    \label{fig:pe-pegrna-plasmids}
\end{figure}

\begin{figure}[ht]
    \centering
    \subfigure[Targetting]{
        \includegraphics[width=0.46\textwidth]{dissertation-prime-editing-process-1.png}
        \label{fig:prime-editor}
    }
    \subfigure[Nicking]{\includegraphics[width=0.46\textwidth]{dissertation-prime-editing-process-2.png}}
    \subfigure[PBS Hybridization]{\includegraphics[width=0.46\textwidth]{dissertation-prime-editing-process-3.png}}
    \subfigure[Reverse Transcription]{\includegraphics[width=0.46\textwidth]{dissertation-prime-editing-process-4.png}}
    \subfigure[Flap Excision]{\includegraphics[width=0.46\textwidth]{dissertation-prime-editing-process-5.png}}
    \subfigure[Completion]{\includegraphics[width=0.46\textwidth]{dissertation-prime-editing-process-6.png}}
    \caption[Process of prime editing]{Process of prime editing after successful expression and transcription of PE and pegRNA plasmids. \textbf{(a)} ngRNA binds to the target DNA loci (protospacer) directly upstream of PAM; \textbf{(b)} SpCas9 introduces nick 3-4 bp upstream of PAM; \textbf{(c)} The now exposed strand hybridizes with the PBS sequence in the pegRNA; \textbf{(d)} The RTT in the pegRNA is reverse transcribed into the target DNA; \textbf{(e)} The edited strand anneals to the non-edited strand; \textbf{(f)} The endogenous repair machinery incorporates the edited strand into the genome.}
    \label{fig:prime-editing-process}
\end{figure}

Before the prime editing can start, the prime editor components need to be delivered into the cell. pegRNA and prime editor proteins are typically delivered separately as plasmids, which are double stranded circular DNA molecules that can be easily replicated and transcribed by the transfected cells. The pegRNA plasmids are transcribed into RNA by the cell's endogenous transcription machinery, and the prime editor proteins are expressed by the cell's ribosome. The two components then assemble into the prime editor complex in the cell's cytoplasm (the fluid inside the cell), ready to locate the target loci and begin the editing process\cite{liudavidr.SearchandreplaceGenomeEditing2019}. 

The prime editing process starts with the denaturing (separation of the two strands) of target DNA loci. This allows ngRNA to bind to the complementary sequence immediately upstream of the protospacer adjacent motifs (PAM) required for successful annealing (binding of two complementary strands of DNA or RNA). The Cas9 nickase can then nick the exposed strand of the target DNA, creating a floating 3' end that can be used as a primer for the reverse transcription process. Finally, the PBS sequence in the pegRNA binds to the floating 3' end, priming the reverse transcription of the RTT.

After the transcription finishes, both the reverse transcribed 3' flap and non-edited 5' flap could anneal to the protospacer, and would result in a equilibrium of 3' and 5' flaps. If the edited 3' flap was excised, target would not be edited and can be processed again with prolonged editing. If the 5' flap was correctly excised, then we move on to the final step where the endogenous cellular repair system permanently incorporates the edited strand into the genome by repairing the mismatched base pairs. Similar to the mechanism exploited by base editing, PE3 and PE5 have an additional sgRNA that guides the PE enzyme to nick the non-edited strand, encouraging the cell to use the edited strand as a template for repair\cite{liudavidr.SearchandreplaceGenomeEditing2019}.

The RTT could be designed to introduce any types of mutations, and the reverse transcription mechanism allows prime editors to change bases far (at most 33bp) from the site of nick, instead of the narrow editing window of base editors. This allows prime editors to have lower constraint on the relative location of the PAM sequence, further increasing the versatility\cite{liuPrimeEditingPrecise2023}.

\section{In-silico Prediction of Prime Editing Outcome}

\label{sec:motivation}

More than 6,000 disorders are known to be caused by various types of mutations in the genome, with around 300 new genetic disorders being discovered each year\cite{petraityteGenomeEditingMedicine2021}. Due to its versatility, prime editing has the potential to correct up to 90\% of these disorder-inducing mutations\cite{kantorCRISPRCas9DNABaseEditing2020}, and the coverage is still increasing with new SpCas9 variants that have less and less PAM sequence constraints\cite{waltonUnconstrainedGenomeTargeting2020}. However, for prime editors to be useful in treating human diseases in clinical setting, it is crucial to have the highest possible accuracy with minimal off-target effects during the edits. 

The efficiency of prime editing can vary greatly across different target sites and pegRNA designs\cite{liudavidr.SearchandreplaceGenomeEditing2019}. Empirical approaches have been used to find the optimal prime editor guide given a target edit, which involves testing a large number of combinations of ngRNA, PBS and RTT sequences repeatedly to find the optimal combination that yields the highest editing efficiency.

This is a laborious and time consuming process even for simpler editing tools such as base editors. With the more complicated process of prime editing, the search space is even larger. The lack of efficient optimization process can significantly limit the practicality in the clinical setting, especially in territories with an existing shortage of medical workers. As a result, in-silico optimization methods are highly desirable and has garnered the interest of many researchers.

Previous in-silico methods can be divided into two catagories: hypothesis driven models that use hard-coded rules to calculate the efficiency of a given edit\cite{hsuPrimeDesignSoftwareRapid2021,hwangPEDesignerPEAnalyzerWebbased2021}, and learning-based models that use machine learning to predict the editing outcome. 
The learning-based methods can be further divided into another two groups: the conventional machine learning methods that use hand-crafted features extracted from the sequence and prime editors\cite{liEasyPrimeMachineLearning2021,koeppelPredictionPrimeEditing2023}, as well as the deep learning methods that use the raw sequence data as part of their input\cite{yuPredictionEfficienciesDiverse2023,kimPredictingEfficiencyPrime2021, mathisPredictingPrimeEditing2023}. 

Each category has their own strengths and weaknesses in performance and efficiency. Hypothesis driven and conventional machine learning algorithms are more efficient to train due to the much smaller input size and the lower complexity of the computation. However, hypothesis based methods struggle to unbiasedly optimize the weights attributed to each feature\cite{liEasyPrimeMachineLearning2021}, while conventional machine learning algorithms loses the DNA context data that can be informative to prime editor optimization. 

The deep learning methods require a much larger dataset to train and are computationally more expensive to train and use. They also need more expertise to design and tune, and are more prone to overfitting due to the large number of parameters. However, their superior performance in the prime editing efficiency prediction task has been convincingly demonstrated by a number of models in recent years.

% TODO: overview of DeepPE, DeepPrime, PRIDICT 1 and 2

\subsection{DeepPE and DeepPrime convolutional neural network}

\begin{figure}
    \centering
    \includegraphics[width=0.7\textwidth]{DeepPE-simplified.png}
    \caption[DeepPE architecture]{DeepPE architecture. The input to the CNN is the one-hot encoded wild type and mutated DNA sequences concatenated with the extension RNA sequence. The output of the CNN is concatenated with the extracted features from the pegRNA sequence and the target site, and fed into the final MLP to predict the editing efficiency. Note that a deep reinforcement learning model was used here instead of a pooling layer to maintain local context information.}
    \label{fig:deeppe}
\end{figure}

DeepPE (\autoref{fig:deeppe}) was one of the earliest attempts at leveraging deep learning to achieve above par performance in predicting prime editing outcomes\cite{kimPredictingEfficiencyPrime2021}. It is a convolutional neural network (CNN) that takes the stacked one-hot encoded wild type and  mutated DNA sequences, the extension RNA sequence as well as a number of features extracted from pegRNA sequence and the target site as input. Despite its simple architecture, DeepPE managed to outperform the conventional machine learning models in terms of Pearson's r and Spearman's R. This indicated that the deep learning model was able to better capture the complex interactions between the sequence data and the editing efficiency compared to using the extracted features alone. It was however very limited in terms of the types of edits it can predict, with the base model only capable of predicting the efficiency of G to C replacement at 5bp upstream of the nick. Additional models were trained to predict other types of edits, and most of them also only support one single specific edit.

\begin{figure}
    \centering
    \includegraphics[width=0.85\textwidth]{DeepPrime-simplified.png}
    \caption[DeepPrime architecture]{DeepPrime architecture, updated from DeepPE. The CNN now uses a more conventional pooling layer, followed by a RNN GRU (Gated Recurrent Unit). The convolutional and recurrent layers take the stacked one-hot encoded wild type and mutated DNA sequences as input, and the output is concatenated with the extracted features processed by a MLP module. The concatenated output is then fed into another MLP to predict the editing efficiency.}
    \label{fig:deepprime}
\end{figure}

DeepPrime is a further attempt made by the same group to improve the performance and capabilities of DeepPE, leveraging the much bigger dataset as well as an improved architecture design\cite{yuPredictionEfficienciesDiverse2023}.
Noticing that stacking the additional PBS-RTT sequence did not improve the performance of DeepPE and added further constraint on pegRNA design (CNN only accept fixed length input), DeepPrime only used the wild type and mutated sequences as input to the deep learning model, and added a separate MLP (Multi Layer Perceptron) to process the computed features instead of directly concatenated the features onto the CNN output as DeepPE did. The output of the MLP was then concatenated with the output of the CNN that processed the sequence data, and fed into the final MLP to predict the editing efficiency. 

DeepPrime achieved far superior performance to DeepPE, while at the same time, one single model can now predict all types of edits for a cell line and prime editor pair existing in the dataset. This enabled the training process to take advantage of the shared features between different edits (further enhancing performance) and significantly improve the ease of use.

\subsection{PRIDICT 1 and 2 bidirectional GRU}

At around the same time DeepPrime was published, the PRIDICT model was developed by Mathis et al from the Schwank lab at the University of Zurch, and used RNN (Recurrent Neural Network) instead of CNN to process the sequence data\cite{mathisPredictingPrimeEditing2023}. The model uses a bidirectional GRU (Gated Recurrent Unit) to process the sequence data, whose output is pooled by a pair of global (whole sequence) and local (RTT region) attention(\autoref{fig:pridict}). Similar to DeepPrime, the features are encoded using a MLP and the output of the two models are then concatenated and fed into the final MLP to predict the editing efficiency.

\begin{figure}
    \centering
    \includegraphics[width=0.85\textwidth]{pridict-simplified.png}
    \caption[PRIDICT architecture]{PRIDICT architecture. The input to the RNN is the one-hot encoded wild type and mutated DNA sequences, stacked with RTT, PBS, Protospacer(only for wild type) annotations. The annotation sequences are boolean sequences that indicate the functionality of each nucleotide in the sequence. The output of the RNN is processed by two attention heads, a global one that process the entire sequence and a local one that masks out all regions other than RTT. }
    \label{fig:pridict}
\end{figure}

This RNN based model was able to predict the efficiency of prime editing outcomes with up to 0.8 Pearson's r on their dataset, and was reported to be on par with the performance of DeepPrime.

Spurred by the success of the bidirectional RNN, the PRIDICT 2 model was developed by applying minor tweaks to the architecture as well as the data preprocessing steps and achieved even higher performance\cite{mathisMachineLearningPrediction2024}. More importantly, thanks to the far more diverse edits in the dataset with much longer inserts and deletes than the up to 3bp long edits in DeepPrime, PRIDICTv2 was more capable in the laboratory setting where insertion and deletions are reaching hundreds of base pairs in length\cite{liuPrimeEditingPrecise2023}.

\section{Study Objective}

(Rewrite)

The development of these models inspired me to further explore the potential of deep learning models in predicting prime editing outcomes. In this study, I aim to develop a deep learning model that is on par or even better than the existing models in predicting prime editing outcomes for a wide range of cell lines and edit types. 

I will also explore the possibility of using ensemble learning to improve the performance of the models, and investigate the potential of using the model to predict the efficiency of prime editing outcomes for cell lines and edit types that are not present in the training dataset. 

The best performing trained model would be presented as a web application that can be used by researchers to predict the efficiency of prime editing outcomes for their own cell lines and edit types.
\chapter{Method}

\minitoc

\section{Curating a Benchmarking Dataset}

Before starting the development of novel models, to establish a common ground for evaluating the performance of machine learning models, a benchmarking dataset was curated using data from various literatures. The main data source is the 2024 study by Mathis et al. which has the most diverse editing types ranging from 1-to-5bp replacement to 1-to-15bp deletions and insertions\cite{mathisMachineLearningPrediction2024}. It is complimented by the DeepPrime dataset, as it has a much bigger HEK293T dataset ($\sim290,000$ vs $\sim22,000$ for DeepPrime and PRIDICT respectively) as well as a wider range of cell type and PE pairs that can be used for fine-tuning the model to achieve better generalizability and higher impact\cite{yuPredictionEfficienciesDiverse2023}. 

In the DeepPrime data, the mutated sequences in the datasets were masked, exposing only the region corresponding to PBS-RTT. To match the PRIDICT format, the mutated sequences were unmasked and aligned to the wild type sequences. 

The dataset was parsed to uniform format to preserve all essential information required by the models while also limiting the number of fields for better readability and portability. At the same time, the standardized format allows easier conversion between datasets, with each data source requiring only two parsing functions, one to convert the model dataset to standard format and one to convert from standard format to the format required by the model. 

The standard (std) format contains the following fields:

\begin{itemize}[itemsep=-0mm]
    \item \textcolor{blue}{cell-line}: the cell line used in the experiment
    \item \textcolor{blue}{group-id}: the id of the target loci, used for grouping the data during cross validation
    \item \textcolor{blue}{mut-type}: the type of mutation introduced by the prime editor, 0 for replacements, 1 for insertions, 2 for deletions
    \item \textcolor{blue}{wt-sequence}: the 100bp/74bp long wild type target sequence starting from 10bp/4bp upstream of the protospacer(PRIDCT/DeepPrime dataset)
    \item \textcolor{blue}{mut-sequence}: the 100bp/74bp long edited target sequence starting from 4bp upstream of the protospacer(PRIDCT/DeepPrime dataset)
    \item \textcolor{blue}{protospacer-location}: the location of protospacer sequence complementary to sgRNA in the wild-type sequence 
    \item \textcolor{blue}{pbs-location}: the prime binding site
    \item \textcolor{blue}{rtt-location-wt/rtt-location-mut}: the location of reverse transcription template in the wild-type/edited sequence
    \item \textcolor{blue}{lha-location}: the location of left homology arm
    \item \textcolor{blue}{rha-location-wt/rha-location-mut}: the location of right homology arm in the wild-type/edited sequence
    \item \textcolor{blue}{spcas9-score}: precalculated SpCas9 score
    \item \textcolor{blue}{editing-result}: empirically observed editing efficiency
\end{itemize}

The locations are separated into two columns in the format of "start:end", where start is the 0-based index of the first base of the sequence, while end is for the location of the last base (non-inclusive). All sequences were read in the direction of from protospacer to PAM for easier interpretation. This is often different from previous studies that read from 5' end to the 3' end of the pegRNA.

The data files are named as ``\{format\}-\{source\}-\{cell-line\}-\{PE-version\}.csv". `format' can be `std' for standard format, `shap' for files with only extracted features, as well as formats for individual models containing all the data required for training, such as `pd' for PRIDICT and `dp' for DeepPrime. `source' is the name of the study that the data was extracted from, while `cell-line' and `PE-version' are self explanatory.

Five main datasets of size greater than 10,000 were used frequently for benchmarking in this study: DeepPrime HEK293T PE2, PRIDICT2.0 HEK293T, PRIDICT2.0 K562, PRIDICT2.0 K562MLH1d, and PRIDICT2.0 Adv. The HEK293T cells refer to the human embryonic kidney cells\cite{kavsanImmortalizedCellsOne2011}; `Adv' refers to prime editing performed in-vivo (in a living organism) in mouse liver cells; K562 cells were derived from a chronic myelogenous leukemia patient\cite{lozzioMultipotentialLeukemiaCell1981} (K562MLH1d refers to K562 cells with MMR inhibition using the MLH1dn proteins). Other smaller datasets were used for fine tuning models trained with the DeepPrime HEK293T PE2 dataset, such as the DeepPrime HEK293T PE2-Max dataset.

The editing efficiencies were calculated using the method suggested by Kim et al.\cite{kimPredictingEfficiencyPrime2021} in \autoref{eq:efficiency}, which was also used by PRIDICT:
\begin{equation}
    \label{eq:efficiency}
    \begin{split}
        \text{Editing Efficiency} =& \frac{\text{Edited Count}}{\text{Total Count}} \times 100\% \\
    \end{split}
\end{equation}
Where:
\begin{equation}
    \begin{split}
        \text{Edited Count} =& \text{Read Counts With Intended Edit at Target Site} - \\
        &( \text{Total Read Counts} \times \text{Background Editing  Rate} ) \\
        \text{Total Count} =& \text{Total Read Counts} - (\text{Total Read Counts} \times \\ &\text{Background Editing  Rate})
    \end{split}
\end{equation}
        
Background editing rate refers to the percentage of target loci that were edited without being transfected with the prime editor. Subtracting the background edits from the total read counts gives the estimated true number of reads that were edited by the prime editor.

The datasets are split into 5 folds based on the group id assigned to each target loci and pegRNA combination. Edits on the same target loci have the same group id, and are thus placed in the same fold to prevent data leakage. The folds are then used for cross validation to evaluate the model's performance.



\section{Data Engineering}
\label{sec:data-engineering}

A conspicuous issue with the dataset is the imbalance in target values. Shown in Figure \ref{fig:imbalanced-original}, the distribution of the editing efficiency is heavily skewed towards the lower end, with a large number of pegRNAs having an efficiency of around 0. This can cause the model to be less accurate in predicting the higher efficiency pegRNAs, as the model would be more inclined to predict lower efficiency to minimize the loss. 

\begin{figure}
    \centering
    \subfigure[Original]{
        \label{fig:imbalanced-original}
        \includegraphics[width=0.45\textwidth]{editing-efficiency-comparison.png}
    }
    \subfigure[Log Adjusted]{
        \label{fig:imbalanced-log}
        \includegraphics[width=0.45\textwidth]{editing-efficiency-log-adjusted.png}
    }
    \subfigure[Undersampling]{
        \label{fig:imbalanced-undersampling}
        \includegraphics[width=0.45\textwidth]{editing-efficiency-undersample.png}
    }
    \subfigure[Quantile transformation]{
        \label{fig:imbalanced-quantile}
        \includegraphics[width=0.45\textwidth]{editing-efficiency-quantile-transform.png}
    }
    \caption[Target Distribution Imbalance]{The distribution of the editing efficiency in the DeepPrime HEK293T PE2 (DP) dataset, the PRIDICT2.0 HEK293T (PD HEK293T) dataset, the PRIDICT2.0 K562 (PD K562) as well as K562 MMR deficient (PD K562MLH1DN) dataset, and the PRIDICT2.0 Adv (PD ADV) dataset. (a), followed by the distribution after several adjustments (b-d). Due to the significant different in dataset sizes, DeepPrime datasets and PRIDICT datasets used different y-axis scale (left and right for DeepPrime and PRIDICT respectively): 
    \textbf{(a)} the distribution of original editing efficiency in the datasets. editing efficiency on the x axis is limited to [0, 100], with bin size of 10; \textbf{(b)} the distribution of the log adjusted editing efficiency in the DeepPrime HEK293T PE2 dataset; \textbf{(c)} the distribution of the editing efficiency after undersampling the data with editing efficiency $<10$ with a ratio of 10:1 (10\% of the original data were preserved); \textbf{(d)}, the distribution of the editing efficiency after uniformly quantile transformed.}
    \label{fig:imbalanced}
\end{figure}

Although the research for imbalanced regression task is relatively limited compared to classification, a number of methods have been proposed\cite{krawczykLearningImbalancedData2016}. The simplest method is to adjust the target values so that better balance can be achieved in the projected space. Log transformation is a suitable method for the editing efficiency dataset, as it increases the distance between the lower values while keeping the higher values close (Figure \ref{fig:imbalanced-log}). The numpy \verb|log1p| function was used to prevent the transformation from being undefined when the target value is 0. During inference, the predicted values were transformed back to the original scale using the \verb|expm1| function. Undersampling is also a useful technique, by removing the majority of the lower efficiency pegRNAs, the model can be trained to better predict the higher efficiency pegRNAs\cite{torgoResamplingStrategiesRegression2015}. 
The undersampling ratio was set to 10:1, with the majority of the pegRNAs with efficiency $<10$ removed (Figure \ref{fig:imbalanced-undersampling}). The undersampling ratio and threshold for determining the part of the data to undersample was fine tuned in the list of [20:1, 15:1, 10:1, 5:1] and [5, 10, 15] respectively using a XGBoost model trained on the first fold of DeepPrime HEK293T PE2 dataset. 
SMOTE (Synthetic Minority Over-sampling Technique) was also considered, but the generation of synthetic data is very difficult for the prime editing prediction task, as the pegRNA sequence is highly structured and the synthetic data generated may not be biologically feasible. Additionally, quantile transformation was also tested. However, since it discretizes the data, some precision is expected to be lost during the inverse transformation.

In addition to adjusting the target values, the loss function can also be adjusted to assign different importance to different target values. Sample weights can be used to penalize the model more for mispredicting the higher efficiency pegRNAs. 

DeepPrime applied a weighted loss function of the editing efficiency tuned for their dataset:
\begin{equation}
    \text{weight} = \text{min}(\exp(6(\log(x+1)-3)+1),5)
\end{equation}
where x is the measured editing efficiency. The weight is then used in the loss function to penalize the model more for mispredicting the higher efficiency pegRNAs.

At the mean time, PRIDICT did not report any special treatment for the imbalanced dataset, possibly because the imbalance is less severe in their dataset (Figure \ref{fig:imbalanced-original}, the PRIDICT HEK293T dataset has a more balanced distribution with large amount of examples with high editing efficiency). 

To verify if the adjustments had made any significant different on the model's performance, a simple two layer MLP with 128 hidden units in each layer was trained on the DeepPrime HEK293T PE2 dataset using the original, log adjusted, undersampled and quantile transformed datasets. An additional model using the DeepPrime adjusted MSE was also tested. The models were trained using the Adam optimizer with a learning rate of 0.005, and the learning rate was adjusted using the CosineAnnealingLRWarmRestart scheduler, with a T\_0 (epoch of the first restart) of 15 and T\_mult (the factor to extend the restart intervals) of 1 to escape local minima. 

\begin{figure}
    \centering
    \vspace{-3mm} % Reduce vertical space between subfigures
    \subfigure[DP HEK293T]{
        \centering
        \includegraphics[width=0.8\textwidth]{adjustment-dp-hek293t-performance.png}
        \label{fig:dp-hek293t-adjustment-performance}
    }
    \vspace{-3mm}
    \subfigure[PD HEK293T]{
        \centering
        \includegraphics[width=0.8\textwidth]{adjustment-pd-hek293t-performance.png}
        \label{fig:pd-hek293t-adjustment-performance}
    }
    \vspace{-3mm}
    \subfigure[PD K562]{
        \centering
        \includegraphics[width=0.8\textwidth]{adjustment-pd-k562-performance.png}
        \label{fig:pd-k562-adjustment-performance}
    }
    \vspace{-3mm}
    \subfigure[PD K562MLH1dn]{
        \centering
        \includegraphics[width=0.8\textwidth]{adjustment-pd-k562mlh1d-performance.png}
        \label{fig:pd-k562mlh1d-adjustment-performance}
    }
    \vspace{-3mm}
    \subfigure[PD Adv]{
        \centering
        \includegraphics[width=0.8\textwidth]{adjustment-pd-adv-performance.png}
        \label{fig:pd-adv-adjustment-performance}
    }
    \caption[DeepPrime model performance comparison after adjustments]{Performance comparison of the DeepPrime model trained on the DeepPrime HEK293T PE2 dataset. The asterisks indicating significance are only shown when an adjustment performed significantly \textbf{better} than training on original data. \textbf{(a)}, the PRIDICT2.0 HEK293T dataset \textbf{(b)}, the PRIDICT2.0 K562 dataset (PD K562\textbf{(c)}), the PRIDICT2.0 K562MLH1d dataset \textbf{(d)}, and the PRIDICT2.0 Adv dataset \textbf{(e)} using the log adjusted (LA), undersampled (US), quantile transformed (QT), and DeepPrime weighted MSE (WMSE) adjustments, in addition to the original dataset and citerion (OG). 
    The performance of each fold is shown as a strip plot on top of the matching bar.}
    \label{fig:adjustment-performance}
\end{figure}

The result of the five adjustments on the five datasets are shown in \autoref{fig:adjustment-performance}. Significant improvements were only observed in three occasions ($p<0.05$, paired t-test between performance in each fold). The log adjustment improved both the Pearson ($p<0.001$) and Spearman ($p<0.01$) correlation of the model on the DeepPrime HEK293T PE2 dataset, as well as the Pearson correlation on the PRIDICT K562 dataset ($p<0.001$), while undersampling noticeably improved both correlation coefficients on the PRIDICT Adv dataset ($p<0.001$). However, the improvement is not universal, as significantly lower performance for undersampling was observed on all datasets other than PRIDICT Adv ($p<0.001$). At the same time, log adjustment resulted in severely reduced performance PRIDICT HEK293T and K562MLH1dn datasets ($p<0.001$). 

This narrows down the option to log adjustment and undersampling. numpy provides a very efficient implementation of the log1p and expm1 functions, making the log adjustment a very attractive option. Meanwhile, the undersampling method is even simpler to implement, and it has the added benefit of reducing the training time, as the model would have to process less data. However, on more complex models, the undersampling method significantly reduce the training size, which may lead to overfitting and reduced validation performance.


% reduce subfigure vertical space
\begin{figure}
    \centering
    \vspace{-3mm} % Reduce vertical space between subfigures
    \subfigure[DP HEK293T]{
        \includegraphics[width=0.8\textwidth]{adjustment-deepprime-dp-hek293t-performance.png}
        \label{fig:deepprime-dp-hek293t-adjustment-performance}
    }
    \vspace{-3mm} % Reduce vertical space between subfigures
    \subfigure[PD HEK293T]{
        \centering
        \includegraphics[width=0.8\textwidth]{adjustment-deepprime-pd-hek293t-performance.png}
        \label{fig:deepprime-pd-hek293t-adjustment-performance}
    }
    \vspace{-3mm} 
    \subfigure[PD K562]{
        \centering
        \includegraphics[width=0.8\textwidth]{adjustment-deepprime-pd-k562-performance.png}
        \label{fig:deepprime-pd-k562-adjustment-performance}
    }
    \vspace{-3mm} 
    \subfigure[PD K562MLH1dn]{
        \centering
        \includegraphics[width=0.8\textwidth]{adjustment-deepprime-pd-k562mlh1d-performance.png}
        \label{fig:deepprime-pd-k562mlh1d-adjustment-performance}
    }
    \vspace{-3mm} 
    \subfigure[PD Adv]{
        \centering
        \includegraphics[width=0.8\textwidth]{adjustment-deepprime-pd-adv-performance.png}
        \label{fig:deepprime-pd-adv-adjustment-performance}
    }
    \caption[DeepPrime model performance comparison after adjustments]{
        Similar to \autoref{fig:adjustment-performance}, the performance comparison of the MLP model trained on the DeepPrime HEK293T PE2 dataset \textbf{(a)}, and the PRIDICT2.0 HEK293T dataset \textbf{(b)}, the PRIDICT2.0 K562 dataset \textbf{(c)}, the PRIDICT2.0 K562MLH1d dataset \textbf{(d)}, and the PRIDICT2.0 Adv dataset \textbf{(e)} using the log adjusted (LA), undersampled (US), and the original dataset (OG). 
    }
    \label{fig:deepprime-adjustment-performance}
\end{figure}

To further verify the necessity of the adjustments, the DeepPrime model with log adjusted and undersampled target were trained on the five datasets using the optimized hyperparameters reported by Yu et al in their study (detailed training process to be discussed in \autoref{sec:training-deepprime-pridict})\cite{yuPredictionEfficienciesDiverse2023}. The performance of the models were then evaluated using Pearson and Spearman correlation and compared to the result trained on the original model, shown in \autoref{fig:deepprime-adjustment-performance}.

As expected, the relative performance of the undersampling adjustment took a nosedive when training the DeepPrime model with far more parameters than the simple MLP model. Significantly lower performance was observed on all datasets ($p<0.01$) trained with undersampled training data. The log adjustment, on the other hand, showed a significant improvement in terms of Spearman's $\rho$ on the DeepPrime HEK293T dataset ($p<0.001$), the PRIDICT K562 dataset ($p<0.001$), and the PRIDICT Adv dataset ($p<0.01$). However, at the same time, significant decrease in performance was observed in terms of Pearson's R for the DeepPrime and PRIDICT HEK293T datasets ($p<0.05$). 

Taking into account the performance of the DeepPrime model, I decided to follow PRIDICT's protocol and not adjust the target values. 

Another problem with the dataset is the lack of variety in the PBS length and location in the PRIDICT dataset. For some reason, all PBS sequences start at 5bp downstream of the protospacer start location and have a length of 13bp, ending before the nick site at 3bp upstream of PAM. This can significantly impact the model's performance during inference on data with arbitrary PBS length and location, especially if PBS related features were directly used by the models. 

To help the models trained on PRIDICT datasets work on unseen data supplied by the users, the suggested PBS length is fixed at 13bp when inferencing using models trained on PRIDICT dataset, with PBS always starting at 5bp downstream of the protospacer start location.



\section{Determinants of Prime Editing Outcome}
\label{sec:determinants}

\begin{figure}
    \subfigure[DeepPrime]{
        \label{fig:shap-dp-pe2-hek}
        \includegraphics[width=0.99\textwidth]{shap-dp-hek293t-pe2.png}}
    \subfigure[Deletion]{
        \includegraphics[width=0.33\textwidth]{shap-pd-hek293t-pe2-delete.png}
    }%
    \subfigure[Insertion]{
        \includegraphics[width=0.33\textwidth]{shap-pd-hek293t-pe2-insert.png}
    }%
    \subfigure[Replacement]{
        \includegraphics[width=0.33\textwidth]{shap-pd-hek293t-pe2-replace.png}
    }%
    \caption[SHAP analysis for DeepPrime and PRIDICT2.0 datasets]{SHAP analysis for DeepPrime and PRIDICT2.0 datasets on HEK293T cell line: \textbf{(a)} Top 15 determinants from SHAP analysis using all of DeepPrime HEK293T Datasets. \textbf{(b-d)} Top 10 determinants from SHAP analysis using PRIDICT2.0 HEK293T data for each individual editing types. The colour of the individual data point shows their normalized values from high (red) to low (blue). The binary values have 1 (True) for high and 0 (False) for low. A number of shorthands have been used to improve the clarity of the illustrations. gcc for GC Content (different from GC count); et for editing type; tm for melting temperature; mfe for minimum free energy; n ap \# for base n at protospacer position \#; max cns for the length of maximum consecutive sequence of base n.} 
    \label{fig:shap}
\end{figure}

As discussed in \autoref{sec:motivation}, most of the recent deep learning models surveyed in the literature (DeepPrime\cite{yuPredictionEfficienciesDiverse2023} and PRIDICT\cite{mathisPredictingPrimeEditing2023,mathisMachineLearningPrediction2024}) followed a similar two model structure. The first model is a deep neural network taking in raw sequence data, and the second model is a smaller MLP/regression model that takes as input the features extracted from the PBS, RTT and ngRNA. For single model architecture (DeepPE\cite{kimPredictingEfficiencyPrime2021}), the features are concatenated to the sequence data and fed into the model.

As a result, it is important to find the most prominent determinants of prime editing outcomes and to extract the most informative features for describing the pegRNA. A large number of features were selected from various sources as the studies focus on different aspects of pegRNA. Shapley Additive Explanations (SHAP) with XGBoost regressor to rank the features based on their importance. The full list of features investigated can be found in \autoref{appendix:features}. 

A number of Python libraries were utilized when extracting the features into the `shap' format. The melting temperature of the sequences were calculated using the 'biopython' library\cite{cockBiopythonFreelyAvailable2009}, the minimum free energy was calculated with the ViennaRNA package and adjusted using sequence length to ensure it's mostly influenced by sequence's nucleotide order and composition\cite{lorenzViennaRNAPackage2011,trottaNormalizationMinimumFree2014}, and the SpCas9 scores were provided by the 'DeepSpCas9' model\cite{kimSpCas9ActivityPrediction2019}. Other features such as GC content were extracted using simple Python string processing functions.

SHAP analysis was first conducted on the largest DeepPrime PE2 HEK293T dataset to provide the most robust identification of the most informative features (Figure \ref{fig:shap-dp-pe2-hek}). The features were then sorted based on their importance, and the top 24 features were used for the model, matching the design optimized by DeepPrime\cite{yuPredictionEfficienciesDiverse2023}. To identify the influence of the features on different editing types, SHAP analysis was also conducted on the PRIDICT2.0 HEK293T dataset for each individual editing type (\autoref{fig:shap}(b-d)), due to its higher variety of editing length. 

The major determinants are as follows:
\begin{itemize}[itemsep=-0mm]
    \item \textcolor{red}{Editing type and length}: editing type has a significant impact on the editing efficiency, with the replacement type having higher efficiency than the insertion and deletion, as shown by the `et replacement' feature in Figure \ref{fig:shap-dp-pe2-hek}. The length of the edit also has great influence on editing result, and its important is more pronounced for deletions and insertions than replacements. As expected, longer edits have lower efficiency due to the increased difficulty in the annealing and repair process.
    \item \textcolor{red}{Melting temperature}: melting temperature (Tm) refers to the temperature at which half of the RNA strand become unfolded or denatured. It is strongly correlated with the editing efficiency, possibly due to its influence on RNA structural stability. Its actual effect, however, is highly mixed. A low Tm in extension seems to positively affect the editing result, while the opposite is true for pbs. At the same time, high Tm in RHA has a mixed impact in the DeepPrime dataset (`tm rha' in Figure \ref{fig:shap-dp-pe2-hek}), while in the PRIDICT2\.0 dataset, high Tm in the PBS is clearly beneficial for the editing process of all types (`tm rha' in \autoref{fig:shap}(b-d)).
    \item \textcolor{red}{Minimum free energy}: minimum free energy (MFE) describes the lowest possible energy required for a RNA sequence to stay in a particular form\cite{lorenzViennaRNAPackage2011}. It appears to have similar effect to Tm, with low MFE in the extension being beneficial for the editing efficiency.
    % TODO: Check MFE 
    \item \textcolor{red}{GC content/count}: GC count in the PBS sequence was shown to be the most important feature for the DeepPrime dataset (Figure \ref{fig:shap-dp-pe2-hek}), consistent with the observation made by Liu et al in their 2019 study introducing prime editors\cite{liudavidr.SearchandreplaceGenomeEditing2019}. They expected lower editing efficiency with low GC count in the PBS, due to the energetic requirements of hybridization of the nicked DNA strand to the PBS. GC content also correlates to the minimum free energy and melting temperature, as GC base pairs have stronger hydrogen bond than AU base pairs.
    \item \textcolor{red}{Poly-T sequences}: Shown as `max cts' in \autoref{fig:shap}, the length of the longest consecutive T (poly-T) sequences in the cDNA of the 3'extension and spacer RNA sequences has a clear negative impact on the editing efficiency. The poly-T termination signal, while not causing termination in itself, causes catalytic inactivation and backtracking of Pol III\cite{nielsenMechanismEukaryoticRNA2013}
    \item \textcolor{red}{SpCas9 score}: DeepSpCas9 is a deep learning model that estimates the activity of the SpCas9 protein on a target loci, and has been shown to be a good indicator of prime editing efficiency\cite{kimPredictingEfficiencyPrime2021}. This is likely due the SpCas9's role in the prime editing process, as it is responsible for the initial binding of the pegRNA to the target loci. 
    \item \textcolor{red}{PAM disruption}: it was shown that prime editors can sometimes rebind to the edited sequences and induce unintended edits, lowering the editing efficiency\cite{liudavidr.SearchandreplaceGenomeEditing2019}. Thus, the disruption of the PAM sequence is beneficial for the editing efficiency as it prevents the reannealing of the pegRNA to the edited sequence.
    \item \textcolor{red}{LHA/RHA/PBS length}: the length of rha (right homology arm, RTT overhang) is a significant determinant for all editing types, with a short rha length leading to deceased prime editing efficiency, while the positive effect of a long rha is not as pronounced. This suggests the existence of a minimum threshold for rha length below which the edited strand cannot efficiently anneal to the unedited strand, consistent with the findings of Yu et al, 2023, recommending a rha length of at least 7nt\cite{yuPredictionEfficienciesDiverse2023}. Reported as edit position in some literatures, lha length corresponding to the distance between the PAM sequence and the edit location is also a significant feature for all editing types, especially insertion and deletion. A longer lha length adversely affects the editing efficiency, suggesting that although prime editors have less stringent requirements for PAM locations, editing should still be done close to the protospacer whenever possible. As for PBS length, both a very short and a very long PBS have a negative impact on the editing efficiency, suggesting an optimal range.
    \item \textcolor{red}{Protospacer nucleotide composition}: in Figure \ref{fig:shap-dp-pe2-hek}, the nucleotide compositions from protospacer location 13 to 17 (nicking position) were all shown to have a significant impact. In particular, a guanine (G) at position 16 and a cytosine (C) at position 17
    
    This is consistent with Mathis et al's finding that the nucleotide composition from protospacer position 10 to 20 shows high integrated gradient scores, suggesting that the nucleotide composition in this region is highly influential in determining the editing efficiency\cite{mathisPredictingPrimeEditing2023}.
    % TODO explain why
\end{itemize}

On top of the quantitatively invested sequence based features, higher level features have also been shown to impact prime editing efficiency:
\begin{itemize}[itemsep=-0mm]
    \item \textcolor{red}{pegRNA secondary structure}: the secondary structure of the pegRNA can impact the efficiency by inducing degradation of the 3' extension. The defunct pegRNA can still combine with target loci using its functioning 5' ngRNA, thus lowering the editing efficiency\cite{nelsonEngineeredPegRNAsImprove2022}.
    \item \textcolor{red}{MMR Behaviour}: as briefly mentioned at the beginning of \autoref{sec:prime-editing-process}, the mismatch repair(MMR) system adversely affects the editing efficiency during step (e) of the prime editing process. The MMR system may reject the annealing of the editing strand to the target loci, or it may excise the edited sequence, preferring the original sequence\cite{chenEnhancedPrimeEditing2021}. 
\end{itemize}

These features are encapsulated in the editing cell type and PE version and are thus not explicitly included in the feature list. For example, epegRNAs(engineered pegRNA) are improved version of pegRNA that are designed for protection against 3' erosion with a more robust secondary structure than the original versions\cite{nelsonEngineeredPegRNAsImprove2022}. At the same time, PE4 and PE5 are newer versions of prime editors with MLH1dn proteins that can suppress the MMR system\cite{chenEnhancedPrimeEditing2021}, and the HEK293T cell line has weaker MMR activity than other cell lines such
as HAP1(derived from a chronic myelogenous leukemia patient)\cite{mathisPredictingPrimeEditing2023}. 


\section{Conventional Machine Learning Models}
\label{sec:conventional-ml}

Using the top 24 extracted features, a number of conventional machine learning models were trained to serve as a baseline for the deep learning models. The models include Lasso and Ridge regression, Random Forest Regressor, Gradient Boosted Trees from the XGBoost library, and a Multi-Layer Perceptron.

Their hyperparameter were optimized on one fold of the DeepPrime HEK293T using sklearn's GridSearchCV function, following the protocol used by PRIDICT. The performance of the optimized models were then evaluated using Pearson and Spearman correlation on the DeepPrime and PRIDICT dataset for three different cell types on all 5 folds to provide a more accurate estimation of the models' performance. The MLP model involved the additional Skorch library to allow for the use of scikit-learn's GridSearchCV for hyperparameter optimization on the Pytorch model. The optimized parameters for the models are as follows:

\begin{itemize}[itemsep=-0mm]
    \item \textcolor{blue}{Lasso}: \verb|alpha=0.006158482110660266| (coefficient for penalty term for L1 regularization)
    \item \textcolor{blue}{Ridge}: \verb|alpha=494.17133613238286| (coefficient for penalty term for L2 regularization)
    \item \textcolor{blue}{Random Forest}: \verb|n_estimators=200| (number of trees to build), \verb|max_depth=10| (maximum depth of the trees)
    \item \textcolor{blue}{XGBoost}: \verb|n_estimators=200| (number of boosting rounds, similar to number of trees), \verb|max_depth=5| 
    \item \textcolor{blue}{MLP}: \verb|hidden_layer_sizes=(64, 64)| (two hidden layer with 64 units each), \verb|activation='relu'| (rectified linear unit activation), \verb|solver='adam'| (adaptive moment estimation solver), \verb|lr=0.005| (learning rate)
\end{itemize}

When training the models, the MLP makes a futher validation set in the training data to prevent overfitting and improve the model's generalization. However, since sklearn does not offer a monitor for validation loss, the other models were fitted using the entire training data and trained until convergence.

The models' performance were evaluated using Pearson and Spearman correlation between predicted and measured efficiency, and the result of the conventional models on the four datasets were shown in \autoref{fig:conventional_ml_models_performance}. 

The Lasso and Ridge regression are the worst performing models, with similar performance between the two. This is to be expected, as the Lasso and Ridge regression are simple linear models and only differ in the penalty term used for regularization. The Random Forest model performed significantly better than the linear models, but with a much longer training time. Although both being tree based models, XGBoost's gradient boosted trees have far superiour performance and efficiency ($\sim$300 times faster) compared to the Random Forest model thanks to its GPU support and optimized implementation of the gradient boosting algorithm . The MLP model trails closely behind the XGBoost model in terms of Pearson's r, but significantly outperforms the XGBoost model 

\begin{figure}
    \centering
    \includegraphics[width=0.8\textwidth]{conventional_ml_models_performance.png}
    \caption[Conventional ML model performance comparison]{Performance compaison in terms of Pearson's r (\textbf{left}) and Spearman's R (\textbf{right}) for a number of conventional machine learning models on the DeepPrime HEK293T PE2 dataset and the PRIDICT2.0 datasets for three different cell lines. The models are Multi-Layer Perceptron(MLP), Random Forest Regressor(RF), Lasso, Ridge, and XGBoost. The performance of the models were evaluated using mean of 5-fold cross validation. The heatmap were mapped to the same scale of [0, 1] for easier comparison.}
    \label{fig:conventional_ml_models_performance}
\end{figure}


\section{Improving Performance with Ensemble Learning}

The performance of the deep learning models can be further improved by using ensemble learning. The idea behind ensemble learning is to combine the results of various weaker models through various voting methods to achieve better prediction than any individual model.It is a common practice in machine learning, and is frequently seen in the top performing models in competitions hosted by Kaggle. In fact, most of the models tested in this study so far are already using ensembling learning in some way. Random forrest trains multiple decision trees using a subset of features and samples to alleviates the problem of overfitting. The XGBoost model is another ensemble of multiple decision trees using optimized gradient boosting \cite{chenXGBoostScalableTree2016}, while DeepPrime, PRIDICT and the Transformer model are effectively an ensemble of a deep learning model and a MLP, using another MLP as a meta learning to combine the representation produced by the two base models. 

In this study, a number of common supervised ensemble learning methods were evaluated, including weighted averaging, bagging, and adaptive boosting (AdaBoost). Due to the large number of models used in the ensemble, the initial training and testing were conducted on the same subset of the PRIDICT2.0 HEK293T dataset used for the hyperparameter tuning of the LLM model.

Bagging and  AdaBoost are normally used with estimators of the same type, but in this study, I wanted to utilize the strengths of different models to produce a possibly more accurate prediction. Thus, an additional meta learner similar to the ones used in the deep learning models was added to the ensemble models. The meta learner takes the prediction of the individual models as well as the 24 extracted features as input and produces the final prediction.

To first test their validity in this task before moving on to more sophisticated models, the models making up the ensemble are a selection of the conventional ML methods evaluated in \autoref{sec:conventional-ml}. Methods that achieved significantly higher performance than the individual models were then tested on the deep learning models.

% TODO: error distribution of the models

\subsection{Training DeepPrime and PRIDICT}
\label{sec:training-deepprime-pridict}

To begin with, for using the DeepPrime and PRIDICT models in the ensemble, the models need to be retrained on all datasets used in this study. A training script was written for each model with the help of Skorch wrapper, similar to the one used by the MLP model. The scripts loads the dataset, perform data conversions, and train the model using the hyperparameters optimized by the original authors. The hyperparameters selected by DeepPrime were listed in their study, while the parameters for PRIDICT were acquired by loading the pickled trained model's configuration. 

To verify the correctness of my implementation, I trained the models on their respective datasets and evaluated the performance. The features calculated for their models are listed in (Add appendix reference). The DeepPrime model achieved slightly lower performance than the one reported by Yu et al, with a mean pearson's correlation of 0.75 on the DeepPrime HEK293T PE2 dataset, compared to the 0.77 reported in their study. PRIDICT also achieved slightly lower Spearman's correlation on the PRIDICT HEK293T dataset, with a mean of 0.81 compared to the 0.82 reported by Mathis et al. These differences could be a result of different seed used for train val split used during training on each fold, or specific details of the training process, for example the parameters used by learning rate scheduler or the optimizer. 

After verifying the correctness of the model implementation, to remove the added variance of feature selection, all models were reconfigured to use the top 24 features extracted from the SHAP analysis. For the same reason, the MLP model and embedding was also reconfigured to use the same architecture utilized by PRIDICT so that the only difference between the models is the sequence encoding model. 

On top of the difference in architecture, another significant difference between DeepPrime and PRIDICT is the handling of the input sequence data. Although both yu et al and mathis et al reported the usage of one-hot encoding in their publication, the actual implementations were very different. DeepPrime directly inputted the one-hot encoded sequence into the convolutional network, while PRIDICT used an embedding layer to convert the one-hot encoded sequence into a dense representation. The dense representation is considered very beneficial in the NLP task as it significantly reduces the dimensionality of the input data, and can capture the semantic meaning of the the relationship between the words\cite{goldbergPrimerNeuralNetwork2015}. This can be very helpful in other bioinformatic tasks where a segment of genome sequence is considered as a single word, resulting a large vocabulary\cite{cegliaIdentificationTranscriptionalPrograms2023}. However, since we are working on the nucleotide level of representation, the vocabulary size is very small (4), and the semantic meaning of the nucleotides are not as important as the actual sequence. 

To determine the necessity of the PRIDCT embedding, the DeepPrime model was reconfigured to use the same architecture as the PRIDICT model, with the one-hot encoded sequence being inputted into an embedding layer before being fed into the convolutional network, using an embed size of 4 to match the original input size. The model was then trained and evaluated on the DeepPrime HEK293T PE2 dataset, with no significant differences across five folds (0.74 vs 0.75 for using and not using embedding, respectively, p > 0.1, paired t-test). This suggests that as expected, the embedding layer is not necessary for prediction of prime editing efficiency on the DeepPrime dataset.

Additionally, PRIDICT's input data has additional function annotations denoting the sequence's corresponding section in the pegRNA. These annotations are boolean sequences that indicate the functionality of each nucleotide in the sequence, and are used to guide the local attention mechanism to focus on the relevant parts of the sequence. However, to quantitatively test the annotation would result in significant rewrite of DeepPrime model's architecture. Limited by the time and resources, I decided to perform this comparison on the transformer model to be developed in this study. The transformer model would have a more flexible structure that allows for easy modification of different parts of the model. 

For easier identification, the training models stored in `trained-models' directory also follows a uniform naming convention: `model-\{source\}-\{cell-line\}-\{PE-version\}-\{fold\}.pth/pkl/json'. All 5 folds of the models were used during evaluation, and the results were averaged to produce the final prediction. Additionally, PRIDICT and DeepPrime have overlapping datasets, and when a editor-cell line combination is present in both datasets, both models were used to predict the editing efficiency, and the results were averaged again to produce the predicted efficiency.


\subsection{Error Analysis}

To get an understanding of the suitability of the ensemble learning and establish an expectation of how helpful would ensemble learning be with the given set of models, the error distribution of the individual models were analyzed using the PRIDICT HEK293T dataset after they have fully converged. The prediction of the five folds were combined to form the final prediction of the full dataset.

\begin{figure}
    \centering
    \subfigure[][Error Distribution]{
        \centering
        \includegraphics[width=\textwidth]{error_comparison_dp-hek293t-pe2.png}
        % \caption{Error Distribution of the Individual Models}
        \label{fig:error-distribution}
    }
    \subfigure[][Pearson Correlation]{
        \centering
        \includegraphics[width=0.49\textwidth]{error_correlation_dp-hek293t-pe2_pearson.png}
        % \caption{Pearson Correlation of the Error of the Individual Models}
        \label{fig:pearson-correlation}
    }%
    \subfigure[][Spearman Correlation]{
        \centering
        \includegraphics[width=0.49\textwidth]{error_correlation_dp-hek293t-pe2_spearman.png}
        % \caption{Spearman Correlation of the Error of the Individual Models}
        \label{fig:spearman-correlation}
    }
    \caption[Error Analysis of the Individual Models]{Error Analysis of the Individual Models on the DeepPrime HEK293T dataset. \textbf{(a)} The distribution of the error of the individual models at each example, smoothened using a moving window of size 1000, and downsampled with a ratio of 100:1 for better visibility; \textbf{(b)} The Pearson correlation of the error of the individual models. \textbf{(c)} The Spearman correlation of the error of the individual models.}
    \label{fig:error-analysis}
\end{figure}

Shown in \autoref{fig:error-analysis}, high correlation in terms of both Pearson and Spearman correlation was observed between the individual models($r>0.5$ for all model pairs, and $R>0.5$ for all pairs except Ridge and Lasso regressions with other models). Spearman's correlation was more informative in this case, as the low target values of majority of the DeepPrime HEK293T dataset could easily result in a high Pearson correlation, and we care more about the ranking of the predicted efficiencies in this task when comparing different pegRNA designs.

The ridge and lasso regression models had 100\% correlation, as they are essentially the same model with different penalty terms. This ruled out the necessity of using both models in the ensemble, and only the ridge regression was preserved due to its superior stability, as lasso regression's L1 regularization can result in zeros in coefficients and cause abrupt changes in the prediction. 

DeepPrime and PRIDICT models also had a high correlation, despite significant difference in architecture. This may indicate that the two models are strong enough to fit to the data as accurately as possible, and an ensemble of the two models may not be as beneficial as expected. 

Interestingly, despite both being tree based models, the Random Forest and XGBoost models had a relatively low correlation in terms of error, with a Spearman correlation of 0.7. This suggests that the different ensembling method used by Random Forest and XGBoost have an impact on the prediction, and the two models can complement each other in the ensemble. Thus, the testing ensemble models were selected to be the Ridge regression, Random Forest, XGBoost, and the MLP model.

The same analysis was conducted on all PRIDICT datasets, with consistent results, shown in \autoref{appendix:error-analysis-figures}. However, the correlation between DeepPrime and PRIDICT was weaker on those datasets. Still, given the much longer training time of PRIDICT model, only DeepPrime would be used in the final ensemble model.


\subsection{Ensemble Using Weighted Averaging}

The simplest method of ensembling in a regression task is averaging the predictions of the individual models. The averaging can be weighted to give more importance to the more accurate models. 

A number of methods were common in the literature for determining the weights of the models, including weighting based on model's performance on the training/validation set\cite{fathiImprovingPrecipitationForecasts2019}, using a grid search to find the optimal weights \cite{anandWeightedAverageEnsemble2023}, or direct optimization of the weights using a meta learner. Due to the similarity of the grid search and direct optimization models, only the direct optimization and weighted according to the model's performance were tested in this study. For each fold, the meta learner takes the the prediction of the individual models on the training set as input and the measured efficiency as target. Similar to MLP, an additional validation set was created to prevent overfitting during the optimization. For the performance based weighting, the weights for each model during testing were determined by the model's performance using the normalized Pearson's correlation on the training set. 

Additionally, to provide context to the meta learning, weighted averaging was also tested with extended input data. In addition to the prediction of the individual models, the 24 extracted features were inputted into the meta learner. 

\begin{figure}
    \subfigure[Pearson]{
        \centering
        \includegraphics[width=0.45\textwidth]{ensemble_pearson.pdf}
        \label{fig:ensemble-weighted-mean-pearson}
    }
    \subfigure[Spearman]{
        \centering
        \includegraphics[width=0.45\textwidth]{ensemble_spearman.pdf}
        \label{fig:ensemble-weighted-mean-spearman}
    }
    \caption[Ensemble Model Performance]{Pearson \textbf{(a)} and Spearman \textbf{(b)} Performance of the ensemble models using weighted averaging with different input data on the PRIDICT HEK293T dataset. The base learners used the uniform colour of gray, while each ensemble model is represented by a different colour. Three ensemble techniques were tested: direct optimization using only the predictions from base models (opt), direct optimization supplemented by the extracted features (opt-f), as well as performance based weighting (pwm, pearson weighted mean). A dotted line of matching colour was drawn at each ensemble model's mean performance for easier comparison.}
    \label{fig:ensemble-weighted-mean}
\end{figure}

This resulted in three different ensemble models: weighted averaging with only the prediction of the individual models, weighted averaging with the addition of 24 extracted features, and weighted averaging using only performance (Pearson's $r$) on the training set. All methods were tested on the PRIDICT HEK293T dataset, and their performance compared to the individual base model are shown in \autoref{fig:ensemble-weighted-mean}.

Interestingly, the simplest performance based weighting method achieved the best performance in terms of both Pearson and Spearman correlation, while the direct optimization methods could not outperform the individual models on the test set, possibly due to overfitting on the training set. The performance based weighting method was on par with the best individual models XGBoost and random forest, while significantly outperforming both of them when evaluated using Spearman's correlation ($p<0.05$, paired t-test). 

As a result, the performance based weighting method was selected as part of the ensemble model for further testing on the DeepPrime dataset with the inclusion of the DeepPrime model.

\subsection{Ensemble with Bagging}

Bootstrap aggregating, or bagging, is a method that trains multiple models on different subsets of the training data and averages the predictions to produce the final prediction. Bagging works on the expectation that the models trained on different subsets of the data will have different biases and errors, and the averaging of the predictions will reduce the variance of the final prediction\cite{dongSurveyEnsembleLearning2020}.

Although sklearn provides implementation of bagging, their implementation does not work with regressors of different types. The deep learning models and the conventional ml models also use different input data, further exacerbating the difficulty of using preestablished libraries. Thus, a custom implementation of Bagging was written to handle the specific requirements of this task.

The bagging model has two important parameters, the number of rounds of training to perform and the percentage of the training test to sample for each round. In each round, all base models are trained on a different subset of the training data, thus the total number of base learners is the number of rounds multiplied by the number of base models. Due to computational constraints, these parameters were tuned separately on the PRIDICT HEK293T dataset instead of being tuned with a grid search. 

\begin{figure}
    \centering
    \subfigure[Percentage Tuning Pearson]{
        \includegraphics[width=0.45\textwidth]{ensemble_pearson_bagging_percentage.pdf}
        \label{fig:ensemble-bagging-pearson-percentage}
    }%
    \subfigure[Percentage Tuning Spearman]{
        \includegraphics[width=0.45\textwidth]{ensemble_spearman_bagging_percentage.pdf}
        \label{fig:ensemble-bagging-spearman-percentage}
    }
    \subfigure[Round Tuning Pearson]{
        \includegraphics[width=0.45\textwidth]{ensemble_bagging_pearson_round.png}
        \label{fig:ensemble-bagging-pearson-round}
    }%
    \subfigure[Round Tuning Spearman]{
        \includegraphics[width=0.45\textwidth]{ensemble_bagging_spearman_round.png}
        \label{fig:ensemble-bagging-spearman-round}
    }
    \caption[Ensemble Model Performance]{Pearson \textbf{(a), (c)} and Spearman \textbf{(b), (d)} Performance of the ensemble models using bagging with different hyperparameters on the PRIDICT HEK293T dataset. The percentage of samples tested varied from 0.3 to 0.9 with an increment of 0.2 (bag-0.3 to bag-0.9), while the number of rounds varied from 1 to 15 with increasing increment (bag-1 to bag-15). }
    \label{fig:ensemble-bagging-tuning}
\end{figure}

Starting with the percentage of the sample, by fixing the number of rounds to 3, the percentage of the sample was varied from 0.3 to 0.9 with an increment of 0.2. Shown in Figure \ref{fig:ensemble-bagging-pearson-percentage} and Figure \ref{fig:ensemble-bagging-spearman-percentage}, the performance of the ensemble increases with the percentage of the sample and plateaued at 0.7. 

Additionally, fixing the sample percentage at 0.5, the number of rounds was varied from 1 to 15 with a varying increment mimicking the exponential increase. Similar to the effect of increasing the percentage of the sample, the performance of the ensemble increased with the number of rounds and plateaued at 3. Further increase in the number of rounds did not result in significant improvement in the performance of the ensemble.

\subsection{Ensemble with AdaBoost}
\label{sec:ensemble-adaboost}

Unlike weighted averaging and bagging, AdaBoost is a boosting method that trains multiple models sequentially. It trains a model on the training data, calculates the error of the model, and then trains another model on the same data with the error of the previous model as the weight of the data. This process is repeated until the desired number of regressors are trained, and the final prediction is the weighted sum of the predictions of the individual models. 

The idea is similar to what DeepPrime did when training on their dataset, but applicable to all dataset without needing to carefully tweaking the weight adjustment functions for each editor and cell line.

Having faced similar problem with bagging, AdaBoost was also implemented from scratch. The implementation in this study is based on the implementation of the AdaBoost.RT algorithm, which is a variant of AdaBoost specifically designed for regression tasks \cite{shresthaExperimentsAdaBoostRT2006,solomatineAdaBoostRTBoosting2004}. However, the original algorithm has to be tweaked in a number of places to accommodate the different types of models used in this study. 

Starting with a set of input data $X=\{x_1, x_2, ..., x_n\}$ and the corresponding target values $Y=\{y_1, y_2, ..., y_n\}$, and an integer specifying the number of rounds $T$, defined similarly to the bagging model. A threshold $0<\epsilon < 1$ was also defined to determine if an individual example is correctly predicted by the model.

A set of sample weights $D_t(i)$ and a model weight $\alpha_t$ and an error rate $\epsilon_t$ were associated with each iteration $t\leq T$.
In the original model, the sample weight distribution is initialized as a uniform distribution summing up to one: $D_1(i) = 1/n$. However, in a dataset with large sample size, the uniform distribution can result in a very small weight for each sample, which does not work well with the XGBoost model and would result in NaN values in its prediction. Thus, I modified the distribution to be a uniform distribution with a sum of number of samples in the dataset, $D_1(i) = 1$. Additionally, the AdaBoost model has no separation of error rate and the model weight. However, since Pearson's correlation is the desirable metric in this task, the model weight was set to be the Pearson's $r$ of the model's prediction on the training set, while error rate was used in adjusting the sample weights.

After each iteration, the absolute relative error for each training example is calculated as $e_t(i) = |y_i - \hat{y}_t(i)|/y_i$, where $\hat{y}_t(i)$ is the prediction of the model at iteration $t$. The error is then used to calculate the error rate, indicating the percentage of examples predicted with higher relative error than the threshold $\phi$: $\epsilon_t = \sum_{i=1: e_t(i) > \phi} D_t(i) / n$. This projects the regression problem into a classification problem, where the AdaBoost algorithm was originally designed for.

The sample weights are then updated using the calculated error rate's power and relative error associated with each sample as 
$$
\beta_t = \epsilon_t ^ n
$$
$$
D_{t+1}(i) = D_t(i) \times \begin{cases}
    \beta_t, & \text{if } e_t(i) > \phi \\
    1, & \text{otherwise}
\end{cases}
$$
Where n can be 1 (linear), 2 (quadratic), 3 (cubic) or any other positive integer. Higher value of n would result in the model focusing more on the misclassified examples, which can help the model to better fit the data, but would also increase the risk of overfitting.

To prevent the sample weights from becoming too small for the XGBoost model during training, the sample weights were clipped to a minimum of 0.01. After clipping, the distribution is then normalized to sum up to 1, ensuring that the sample weights are still a probability distribution.

The final output is then calculated as the weighted sum of the predictions of the individual models using the normalized model weights $\alpha$, similar to the performance weighted mean and bagging models.

Tuning for AdaBoost is a lot more complicated, as the error rate threshold has a very significant impact on the performance of the model, and needs to be selected carefully for every set of data. A threshold too low would result in insufficient amount of correctly predicted examples, while a threshold too high would result in overfitting to the outliers with very high relative error\cite{shresthaExperimentsAdaBoostRT2006}. 

Ideally, the tuning process should be repeated for each dataset. However, considering the time and resource limit of the project, the configuration tuned for the PRIDICT HEK293T PE2 dataset was used as a representative for all datasets. 



\section{Adoption of Transformer Model}

% TODO: Add more details about the transformer
The transformer architecture was a ground breaking innovation in the field of NLP, relying solely on self-attention mechanism to create a representation of the input that captures the long range dependencies in the sequences\cite{vaswaniAttentionAllYou2017}. This makes it a suitable candidate for processing the wide target sequences, as the objective of RNN and CNN used by PRIDICT and DeepPrime was to capture the influence of the surrounding nucleotides on the editing efficiency.

It was adopted by the Schwank lab (designers of PRIDICT) in the task of base editing efficiency prediction (BE-DICT), and achieved superior performance in multiple datasets compared to its RNN and CNN counterparts\cite{marquartPredictingBaseEditing2021}. In their task, the transformer model was used to predict the bystander editing efficiency of the base editors given a protospacer location. The bystander editing refers to the phenomenon where the base editor induces unintended edits in the DNA sequence by modifying other targetable nucleotides in the vicinity of the activity window. A number of possible edits result and its corresponding efficiency were inputted into the model during training, and the model can accurately output the efficiency of all possible outcome during inference, even at unseen target loci. Although having slightly lower performance when compared to other models focusing on predicting the intended edits, the transformer model was able to predict the bystander editing efficiency with far superior accuracy.

The success of BE-DICT in predicting the bystander editing efficiency suggests that the transformer model may be capable of capturing the complex relationship between the wild type and edited sequences. This prompted me to utilize this architecture in the task of prime editing efficiency prediction, which involves a similar targeting mechanism as base editing.

\subsection{Model Architecture}

\begin{figure}
    \centering
    \includegraphics[width=0.8\textwidth]{transformer-architecture.png}
    \caption[Transformer Model Architecture]{Transformer Prime Editing Efficiency Prediction Model Architecture. The sequence data is one hot encoded and fed into the transformer model, whose cross attention layer is used to capture the relationship between the wild type and edited sequences and outputs an encoding of both sequences. The encoding is then fed into the MLP header alongside the encoded features to produce the final prediction.}
    \label{fig:transformer-model}
\end{figure}

The implementation of the transformer architecture was adopted from Sasha Rush's implementation of the transformer model in Pytorch\cite{AnnotatedTransformer}, while the MLP encoder and decoder were implemented according to the cleaner design of the DeepPrime model\cite{yuPredictionEfficienciesDiverse2023}.

The transformer model was used to encode both the wild type and mutated sequence, which is slightly unconventional when compared to its usage in other NLP tasks. Starting with positional encoding, the \verb|sine| and \verb|cosine| waves for even and odd dimensions were first applied to the one hot encoded sequence to assign positional information to the input sequences, as attention mechanism does not have any inherent understanding of the order of the input data. 
To help the transformer model better understand the difference between wild type and edited sequences, the two sequences were aligned so that the non edited part of the sequence has matching positional encoding. The missing regions in the edited sequence for deletion and in the wild type sequence for insertion were filled with padding values, and the positional encoding was set to 0 for these regions.

The wild type sequence is then inputted into the transformer encoder, composed of a self-attention layer and a feed forward layer. The self-attention layer is used to capture the relationship between the nucleotides within the sequence, while the positional feed forward layer further transforms the representation and introduces non-linearity with the activation function. 

Similarly, the mutated sequence is inputted into the transformer decoder, starting with the same self attention layer and positional feed forward layer. The output of the encoder and decoder are then processed by the cross attention layer, using the output of the encoder as the query and the output of the decoder as the key and value. After the cross attention layer, the output is processed by another positional feed forward layer, which completes one layer of the decoder. 

The encoder and decoder could be stacked multiple times for the attention mechanism to capture higher level features of the input data. Residual connections and layer normalization were also used after each multi-head attention and positional feed forward layer to stabilize the training process.

The final output of the transformer model is then pooled by a feature embedding attention layer, which assigns different weights to the output of the transformer model based on the importance of the each token, and produces a flattened representation with length equal to the number of embedding dimensions (4 if only only using one hot encoded sequence, 6 if adding the functional annotation of PBS and RTT regions).

At the same time, the extracted features were processed by a separate MLP, which produces a dense representation of the input and is concatenated with the output of the transformer model. 

Finally, the concatenated representation is processed by the MLP header, which produces a real value as the predicted editing efficiency of the input sequences given the pegRNA.



\subsection{Architecture and Hyperparameter Tuning}

A number of architectural and hyperparameter choices were tested during the development of the transformer model. As mentioned in \autoref{sec:ensemble-adaboost}, the tuning of the model should ideally be repeated for each dataset. Limited by available resource, however, tuning was only performed on the PRIDICT2.0 HEK293T for a balance between dataset size and training time.

Similar to PRIDICT, the size of the MLP layers were controlled using three parameters. The first is the mlp\_embed\_factor, which controls the size of the hidden layers in the MLP encoder and decoder. This is followed by the embed dimension of the MLP, which is the size of the output of the MLP embedder processing the feature vector. The last parameter is the number of encoder units inside of the MLP, where a encoder unit is a combination of two linear layers converting the data from embed dimension to embed dimension * mlp\_embed\_factor, followed by dropout and layer normalization.

To reduce the computational cost by testing many parameters at the same times and forming a large parameter grid, the tuning was conducted in two separate stages, with the first stage of tuning focusing on the a number of design decisions of the model architecture. 

To begin with, a number of alternative attention mechanisms were tested, using the scalar dot product attention as a baseline. 

% Describe flash transformer
To alleviate the quadratic computational cost of the scalar dot product attention model, flash attention was included as a more efficient alternative to the self-attention mechanism. Half precision was required by the flash attention to train and test the model, which can significantly reduce memory usage and training time, achieving up to 8 times speed up when compared to single precision (float32)\cite{micikeviciusMixedPrecisionTraining2018}. However, certain small-magnitude gradients may fall out of range of half precision and become zero. This results in NaN values in the model's weights and losses, causing training to fail. As a result, the hugging face accelerator module was adopted to prevent the training from failing. 

Additionally, global attention is not always necessary for lower level of the transformer model\cite{raeTransformersNeedDeep2020}, thus another 
possible improvement to the model is to use local attention for self-attention layers. The local attention only allows the model to attend to a fixed number of tokens within a certain window, reducing the computational cost of the attention mechanism and maybe improving the model's performance by focusing on the most relevant tokens.

However, during testing, neither the flash attention nor the local attention showed any significant improvement in the model's performance.

Even with the hugging face's accelerator module, the mixed precision training still causes around 1 out of 20 training runs to fail due to NaN values in the model's weights and losses, especially in model configurations with poor performances. Additionally, no significant improvement in either training time or performance was observed when using flash attention when training on the network with 1 encoder and decoder layer using 2 attention heads (one head works on CG representation, the other works on AT representation, performance comparison with pearson's $r$ of 0.79 vs 0.79 across 5 repeated runs, $p>0.1$, paired t-test) compared to scalar product attention. On the same configuration of 1 encoder and 1 decoder, the local attention mechanism also did not show any significant improvement in the model's performance (pearson's $r$ of 0.78 vs 0.79 for with and without local attention respectively, $p>0.1$, paired t-test).

Additionally, although having previously established that the usage of embedding does not significantly improve the DeepPrime and PRIDICT models' performance, the influence of using PRIDICT's functional annotation still needs to be investigated. Using the same configuration with local attention enabled, marginal improvement was observed in the model's performance (3 heads used for annotated representation, with 2 heads processing ATGC representation as before, and 1 extra head processing the PBS and RTT representation, pearson's $r$ of 0.79 vs 0.81 for without and with functional annotation, $p>0.05$, paired t-test). This indicated that the functional annotation may provide important contextual information for the transformer model to differential between different edits on the same target loci. Although the improvement was not significant, the improvement was consistent across all 5 repeated runs, and thus the functional annotation was included in the final model.

As a result, the final model was configured with the scalar dot product attention mechanism with local attention used during the self-attention layers, without the usage of flash attention or functional annotation.

\begin{figure}
    \subfigure[1 e/d unit]{
        \includegraphics[width=0.42\textwidth]{transformer-tune-pdropout-num_encoder_units-fixed-at-1-pearson.png}
        \label{fig:transformer-tune-pdropout-num_encoder_units-fixed-at-1-pearson}
    }%
    \subfigure[3 e/d units]{
        \includegraphics[width=0.27\textwidth]{transformer-tune-pdropout-num_encoder_units-fixed-at-3-pearson.png}
        \label{fig:transformer-tune-pdropout-num_encoder_units-fixed-at-3-pearson}
    }%
    \subfigure[5 e/d units]{
        \includegraphics[width=0.27\textwidth]{transformer-tune-pdropout-num_encoder_units-fixed-at-5-pearson.png}
        \label{fig:transformer-tune-pdropout-num_encoder_units-fixed-at-5-pearson}
    }
    \subfigure[1 e/d units]{
        \includegraphics[width=0.42\textwidth]{transformer-tune-pdropout-num_encoder_units-fixed-at-1-spearman.png}
        \label{fig:transformer-tune-pdropout-num_encoder_units-fixed-at-1-spearman}
    }%
    \subfigure[3 e/d units]{
        \includegraphics[width=0.27\textwidth]{transformer-tune-pdropout-num_encoder_units-fixed-at-3-spearman.png}
        \label{fig:transformer-tune-pdropout-num_encoder_units-fixed-at-3-spearman}
    }%
    \subfigure[5 e/d units]{
        \includegraphics[width=0.27\textwidth]{transformer-tune-pdropout-num_encoder_units-fixed-at-5-spearman.png}
        \label{fig:transformer-tune-pdropout-num_encoder_units-fixed-at-5-spearman}
    }
    \caption[Transformer Model Hyperparameter Tuning]{Pearson \textbf{(a-c)} and spearman \textbf{(d-f)} performance of the transformer model with different dropout rates (pdropout), positional feedforward embedding size (mlp\_embed\_dim) and number of encoder/decoder units (e/d unit) on the PRIDICT2.0 HEK293T dataset. The highest pearson and spearman value among all configuration was drawn as a red line for easier comparison. The strip plot on top of the box plot shows the real data points for each run of the model.}
    \label{fig:transformer-tune}
\end{figure}

The next stage focuses on the parameter size of the model. Since the MLP modules were already optimized by Yu at al in their DeepPrime study, the size of the MLP layers were kept constant. The size of the transformer is mostly controlled by twi parameters: the number of encoder and decoder layers, and the mlp embedding size of the positional feedforward layers. Additionally, the dropout rate was also optimized in this stage as it determines the amount of information that is retained during training. A grid search was conducted on the dataset, using the following configuration for each parameter:

\begin{itemize}[itemsep=-0mm]
    \item \textcolor{blue}{Number of Encoder/Decoder Units}: 1, 2, 3
    \item \textcolor{blue}{Positional Feedforward Hidden Unit Size}: 50, 100, 150
    \item \textcolor{blue}{Dropout Rate}: 0.05, 0.1, 0.2, 0.3, 0.5
\end{itemize}

Each configuration was repeated for 3 runs to get a better representation of the model's performance, with the result shown in \autoref{fig:transformer-tune}. 
The effects of dropout rate varied with the number of encoder and decoder units as well as the positional feedforward embedding size. With 100 hidden units, performance consistently increased with dropout rate 0.2 or 0.3, and then started to plummet. With 50 and 150 units, the trend is similar when 1 or 3 encoder/decoder units were used, but with an initial drop in performance when the dropout rate was increased from 0.05 to 0.1. However, when a stack of 5 encoder/decoder units were used, the 150 hidden unit configurations still follows the first increase then decrease trend, while the 50 hidden unit configuration showed a steady decrease in both pearson and spearman correlation as the dropout rate increased.

Overall, the configuration with dropout rate of 0.2, 1 encoder and decoder unit, and positional feedforward hidden unit size of 100 achieved the highest performance in terms of both Pearson and Spearman correlation, and the 3 encoder and decoder units configuration with 0.3 dropout trails closely behind. The performance difference were not significant between the best and the second best configurations (pearson's $r$ of 0.82 vs 0.82, spearman's rank correlation of 0.84 vs 0.84, $p>0.1$, paired t-test). However, the 1 encoder and decoder unit configuration consumes significantly less computational resources and was significantly faster to train. Taking into the practical considerations, the 1 encoder and decoder unit configuration was selected as the final model.

As a result, the final model was configured with 1 encoder and decoder unit, positional feedforward hidden unit size of 100, 3 scalar dot product attention heads, and a dropout rate of 0.2. 

\section{Development of pegRNA Design Web Tool}

To facilitate the use of the developed models in the community, a web tool was developed for users to input their target edit and receive the best pegRNA design as per the model's prediction. The web tool was implemented using the Django framework for easier interaction with Python scripts. 

User input should be in the PRIDICT format of \{at-lesat-100bp\}-(before-edit/after-edit)-\{at-lesat-100bp\}. The user response is then handled by the prediction API endpoint, which would parse the input into original and mutated sequence as well as edit location, type and length. A number of pegRNAs would then be proposed accordingly in the standard data format, which is then converted to the suitable formats for the models. The best performing model would then be used to predict the editing efficiency of the candidate pegRNAs. 

The result is returned as a JSON response and is parsed by the front end as a table of pegRNA candidates with their corresponding editing efficiency. The pegRNA's composition with regard to the wild type sequence can also be visualized by selecting a particular result for easier interpretation.

Other than the model itself, an algorithm for suggesting candidate pegRNAs given a desirable edit was also implemented. Supposing the nick is done at 3bp upstream of the PAM, for the edit to be possible, the PAM sequence must be at most 3bp downstream of the edit start location (in which case LHA length is 0). Additionally, the protospacer must begin with a guanine nucleotide (G), so that the U6 motor used for the transcription of the pegRNA can recognize the sequence and start the expression\cite{hsieh-fengEfficientExpressionMultiple2020}.

These requirements have the most significant constraint on the LHA length, as it always starts at the nick site 3bp upstream of the PAM and must end at the edit location. Thus, its length depends entirely on the choice of the protospacer location. PBS and RHA, on the other hand, are much more flexible in length, as they can be freely extended to their 3' and 5' ends respectively.
However, as discussed in \autoref{sec:determinants}, the length of LHA, RHA and PBS all have significant impact on the editing efficiency. To come up with a recommended range for these parameters, a more detailed analysis of the relationship between the lengths and the editing efficiency was conducted.

\begin{figure}
    \includegraphics[width=\textwidth]{lha-rha-pbs-length-requirement.png}
    \caption[Relationship between LHA, RHA and PBS length and editing efficiency]{Relationship between PBS (top), RHA (middle) and LHA (bottom) length and editing efficiency. Data from all of the DeepPrime and PRIDICT datasets was used in the analysis for better generalizability. Random samples of 5\% of the individual data pointso were plotted as a strip plot on top of the box plot to shw real data while keeping the figure from clutering. Y axis is limited to a max value of 5 + 1.5 * IQR to further improve readability. }
    % The values associated with rha length over 30bp were omitted due to the small number of data points and very low efficiency.
    \label{fig:lha-rha-pbs-length}
\end{figure}


Shown in \autoref{fig:lha-rha-pbs-length}, the relationship between the lengths of LHA, RHA and PBS and the editing efficiency was analyzed using the all data in the DeepPime and PRIDICT datasets. Consistent with Yu et al, 2023's finding, editing editing efficiency plateaued with RHA length of 7-12bp, and is steadily high with a number of peaks until 20bp long, when the efficiency starts to decrease as the RHA grows.  As for LHA, the efficiency was shown to decrease with longer LHA length, with an elbow point at around 13bp, where the decrease in performance with regard to the lha length speeded up. Thus, the recommended distance between the nick and the edit location is set to 0-13bp, and this is only extended if no suitable PAM is found within the range. Last but not least, as expected, PBS length's impact on the editing efficiency is roughly normally distributed, with the highest efficiency at around 12bp. Decent performance can be observed from 8bp all the way up to the highest recorded value of 17bp, which covers the entire range of the protospacer upstream of the nick site. Thus, although the SHAP analysis showed that long PBS can be detrimental to the editing efficiency, the range of PBS length was still set to 8-17bp.

\begin{table}[ht]
    \centering
    \begin{tabular}{c|c|c}
        % \hline
        \textbf{Component} & \textbf{Min Length} & \textbf{Max Length} \\
        \hline
        LHA & 0 & 13 \\
        RHA & 7 & 20 \\
        PBS & 8 & 17 \\
        % \hline
    \end{tabular}
    \caption{Recommended range for LHA, RHA and PBS length}
    \label{tab:recommended-range}
\end{table}

In summary, unless specified otherwise by the users, the algorithm would only propose pegRNAs with component length in the range specified in \autoref{tab:recommended-range}. The recommending range would result in a maximum of 585 pegRNAs (true number is usually much smaller as LHA location is heavily constrained), which would then be processed and evaluated by the model to produce the estimated editing efficiency.


Limited by the scale and funding of this project, the web application is not deployed to a public server at the current stage. However, it can still be tested locally using Python with Django and Pytorch installed by executing:

\verb|python webtool/manage.py runserver|

suppose that the user is in the root directory of the project. The web tool can then be accessed by visiting \verb|http://127.0.0.1:8000/|.

\begin{figure}
    \centering
    \includegraphics[width=\textwidth]{webapp-interface.png}
    \caption[pegRNA Design Web Tool Interface]{Interface of the pegRNA Design Web Tool. A dropdown menu is provided for the user to select the PE version and cell line, and the user can input their target edit in the correct format in the input box.}
    \label{fig:webtool}
\end{figure}

The user can supply their target edit in the correct format in the input box, and can also uses the test example by clicking the `Use Test Example' button. The PE and Cell Line should also be selected from the dropdown menu above the text box before running the prediction, shown in \autoref{fig:webtool}. Any errors in the input would be communicated to the users with browser notifications. 

\begin{figure}
    \centering
    \includegraphics[width=0.6\textwidth]{webapp-table.png}
    \caption[pegRNA Design Web Tool Output]{Output of the pegRNA Design Web Tool. The output contains the position of the protospacer, RTT and PBS, as well as the predicted editing efficiency. For each pegRNA, the user can click on the `Visualize Sequence' button to see the composition of the pegRNA with regard to the wild type sequence.}
    \label{fig:webtool-output}
\end{figure}

\begin{figure}
    \centering
    \includegraphics[width=\textwidth]{webapp-visualization.png}
    \caption[pegRNA Design Web Tool Visualization]{Visualization of the pegRNA Design Web Tool. The composition of the pegRNA with regard to the wild type sequence is shown, with the brackets indicating the position of the protospacer, RTT and PBS.}
    \label{fig:webtool-visualize}
\end{figure}

% TODO replace all figures in this section

After the user submits the input, the web tool would process the input and return the a number of proposed pegRNAs in a table format (\autoref{fig:webtool-output}). The output contains the position of the protospacer, RTT and PBS, as well as the predicted editing efficiency. For each pegRNA, the user can click on the `Visualize Sequence' button to see the composition of the pegRNA with regard to the wild type sequence (\autoref{fig:webtool-visualize}).
\chapter{Benchmarking and Result}

\section{Fine Tuning the Deep Learning Models}

On top of the five main dataset of size $> 20k$, DeepPrime has additional 18 datasets of size $\sim5k$, which were used to fine tune the model for better generalizability. As a result, for a more thorough comparison between the models, the same fine tuning process was conducted on all deep learning models.

To fine tune the models, the same hyperparameters during the original training were used, except for the learning rate, which was lowered to $1e-3$ for the fine tuning to accommodate the smaller dataset size. Additionally, the NRCH versions of prime editors were used in some of the small datasets, which contains optimized SpCas9 proteins that can accommodate NRCH PAMs (N = A, T, C, G; R = A, G; H = A, C, T) \cite{millerContinuousEvolutionSpCas92020}. The pam matching function was thus updated to accommodate the additional PAMs when determining if the protospacer adjacent motifs were disrupted during the edit.

To preserve the sequence features acquired on the DeepPrime main dataset, the sequence processing layers were frozen, leaving only the feature processing and the final meta learner to be trained. The models were trained for at most 300 epochs for each fold, and the best model was selected based on the validation loss.

% TODO: add references to the section and appendix after including the results

\section{Performance on All Datasets}

\begin{figure}
    \centering
    \includegraphics[width=\textwidth]{all_models_performance.png}
    \caption[Performance of the models on all datasets]{Performance of the deep learning models on all datasets. The x-axis represents the model name, while the y-axis represents the datasets used for training and testing. The mean performance of the models across five folds is shown in the heatmap, with brighter colours indicating higher pearson or spearman correlation.
    As mentioned in \autoref{sec:datasets}, the datasets are named in the format of `data source - cell line - prime editor'. 
    
    There were three data sources involved during benchmarking - DeepPrime (dp), PRIDICT (pd), and DeepPrime Small (dp\_small, $\sim5k$ examples). At the same time, the cell line and prime editors' names are self-explanatory. 
    
    Four models in total were produced in this study - the transformer based model (tr), the power weighted mean ensemble (pwm), the bagging ensemble (bag), and the adaboost ensemble (ada). Their performances are highlighted in the heatmap with the red bounding box.
    They were compared with the two state of the art deep learning solutions, DeepPrime (dp) and PRIDICT (pd). The datasets where any of the models trained in this study significantly outperformed both DeepPrime and PRIDICT are marked with a blue bounding box (performance across five folds, $p<0.05$, paired t-test).
    }
    \label{fig:performance}
\end{figure}

As discussed in \autoref{sec:ensemble}, all three ensemble models were able to significantly outperform the base learners on the PRIDICT HEK293T PE2 dataset. Thus, power weighted mean ensemble, bagging and adaboost were trained on all available datasets alongside the transformer model to evaluate their performance against the base learners, as well as DeepPrime and PRIDICT.

Unfortunately, the transformer model failed to significantly outperform DeepPrime and PRIDICT on any of five main datasets, with lower Pearson's $r$ and similar Spearman's $\rho$. Additionally, when compared with the conventional machine learning models' performance in \autoref{fig:conventional_ml_models_performance}, only marginal improvement was observed over the MLP model that makes up part of the transformer model in the DeepPrime dataset ($r$ of 0.69 vs 0.67) and no improvement in the PRIDICT dataset. This indicates that the transformer model may not have contributed significantly to the outcome, and the performance was mainly driven by the MLP model. 

However, it did outperform DeepPrime and PRIDICT on two of the DeepPrime small datasets (A549 PE2-max and A546 with PE2-max using engineered pegRNA (epegRNA)) in terms of Spearman correlation, but not Pearson correlation. The consistently better performance on A549 PE2-max dataset may suggest that the motifs learned by the transformer model is especially useful for this configuration. 

The three ensemble methods showed similar performance with each other across all datasets, with bagging and adaboost often marginally outperforming the power weighted mean ensemble. They significantly outperformed PRIDICT and DeepPrime on two of the PRIDICT datasets in terms of Pearson or Spearman correlation (HEK293T PE2 and K562MLH1dn PE2), as well as on two of the DeepPrime small datasets. One of the most significant boost in performance was observed on the PRIDICT HEK293T PE2 dataset, where the tuning of AdaBoost and Bagging ensemble models' hyperparameters was performed. This hints that the ensemble methods, especially AdaBoost and Bagging, can be sensitive to the dataset size and composition, and may benefit from retuning on each dataset.


\section{Performance on Individual Editing Types}

On top of the overall performance, the performances of the models on individual editing types were also evaluated on the five main datasets to understand the models' ability to handle different types of mutations. 

\begin{figure}
    \centering
    \includegraphics[width=0.8\textwidth]{all_models_performance-edits.png}
    \caption[Performance of the models on individual editing types]{Performance of the deep learning and ensemble models on individual editing types. Contrary to \autoref{fig:performance}, the x-axis now illustrates the name of the dataset and the corresponding edit type subset (sub[stitution], ins[ertion], [deletion]), while the y-axis represents the model names. Similar to \autoref{fig:performance}, the mean performance of the models across five folds is shown in the heatmap, with brighter colours indicating higher pearson or spearman correlation. The only dataset where the models trained in this study achieved statistically significant improvement over DeepPrime and PRIDICT was the PRIDICT HEK293T PE2 dataset, marked with a blue bounding box.}
    \label{fig:performance_edits}
\end{figure}

Illustrated in \autoref{fig:performance_edits}, the performances of predicting individual edits were mostly consistent with the overall performances in terms of Pearson's $r$. The transformer model falls short of the performance of PRIDICT and DeepPrime on all edit types of all datasets other than PRIDICT HEK293T PE2, where it marginally outperformed PRIDICT on insertion examples.

At the same time, although the transformer model was on par with PRIDICT and DeepPrime when evaluated with Spearman's $\rho$ on overall data, the relative performances of the models were not consistent across different mutations. The transformer model was often on par or can marginally outperform PRIDICT and DeepPrime when predicting insertion and deletion efficiency. However, the performance of the transformer model on substitution examples was consistently lower than PRIDICT and DeepPrime.

The relative performances of the ensemble models with DeepPrime and PRIDICT were more complex and varies more substantially across different datasets, possibly due to the sensitivity to hyperparameters mentioned in the previous section. 

The ensemble models had slightly lower Pearson and Spearman performance on the substitution and insertion examples of the PRIDICT K562 dataset, but noticeably outperformed PRIDICT and DeepPrime on the deletion examples. The same was true for the PRIDICT Adv dataset when evaluated with Spearman's $\rho$, while only similar performance for the deletion examples was observed when evaluated with Pearson's $r$.

Pronounced improvement on the deletion examples was also observed on the PRIDICT K562MLH1dn PE2 dataset, but in this case, the ensemble models were able to match the performance of PRIDICT and DeepPrime on the substitution and insertion examples.

On the PRIDICT HEK293T PE2 dataset, the ensemble models significantly outperformed PRIDICT and DeepPrime on all edit types ($p<0.05$, paired t-test across five folds), while the opposite was true for the larger DeepPrime dataset.

\section{Attention Analysis}
\label{sec:attention_analysis}

\begin{figure}
    \centering
    \subfigure[Substitution]{
        \includegraphics[width=0.85\textwidth]{dp-hek293t-pe2-transformer-only-attention-replace-1bp.png}
        \label{fig:attention_insertion}
    }
    \subfigure[Insertion]{
        \includegraphics[width=0.85\textwidth]{dp-hek293t-pe2-transformer-only-attention-insertion-1bp.png}
        \label{fig:attention_substitution}
    }
    \subfigure[Deletion]{
        \includegraphics[width=0.85\textwidth]{dp-hek293t-pe2-transformer-only-attention-deletion-1bp.png}
        \label{fig:attention_deletion}
    }
    \caption[Attention weights for the DeepPrime model trained on the HEK293T-PE2 dataset]{Attention weights of length 1 edits for the DeepPrime model trained on the HEK293T-PE2 dataset. The x-axis represents the position of the transformer output token, the y-axis indicates the edit position. Red colour indicates higher attention weights, while blue represents lower attention weights. The example size for each edit location is shown on the right in the format of `n=number of examples'.}
    \label{fig:attention}
\end{figure}

A significant advantage of the attention based methods are their interpretability. The attention mechanism allows the model to focus on specific parts of the input data using the attention weights, which can be visualized teo understand some of the model's decision making process. 

In this architecture, the most informative and interpretable attention weights are the feature embedding attention weights at the final layer of the transformer model, pooling all token embeddings into a single feature embedding by attributing different weights to each token position. 

For better clarity, the examples were grouped together by their editing type and length so that the edit positions can be easily identified and compared. And the attention weights for edits at the same position were aggregated to show the overall importance of the position in the editing efficiency prediction.

The model trained on the biggest DeepPrime dataset was first tested, as the transformer model has the greatest chance of learning the underlying motifs influencing the editing efficiency in the data. Starting with the edits of length 1, the attention weights were visualized for the substitution, insertion, and deletion, shown in \autoref{fig:attention}. 

For both 1-bp substitution and insertion, the attention weights were the highest from around the start of the protospacer location (location 4) to the nick position (location 20, also where the PBS ends). This may be an indication that the model considers the composition of the PBS as an important feature for the editing efficiency prediction. Additionally, one of the hot spots for the attention weights was the protospacer location 13 to 17. This is consistent with the finding in \autoref{fig:shap} that the composition of the protospacer is an important feature for the editing efficiency prediction.

However, for deletion, the attention weights were the highest at the edit position. The cross attention output of the edit position for deletion was dominated by the encoder output, as the self-attention weights for the mutated sequence at edit position were masked out to be zero. This is a possible indication that the model considers the base to delete as an important feature for the deletion operation.

\begin{figure}
    \centering
    \includegraphics[width=0.9\textwidth]{dp-hek293t-pe2-transformer-only-attention-deletion-3bp.png}
    \caption[Attention weights for 3-bp deletion of the DeepPrime model trained on the HEK293T-PE2 dataset]{Similar to \autoref{fig:attention}, attention weights of length 3 deletion for the DeepPrime model trained on the HEK293T-PE2 dataset.}
    \label{fig:attention_deletion_3bp}
\end{figure}

\begin{figure}
    \subfigure[Replacement]{
        \includegraphics[width=0.33\textwidth]{shap_1bp-dp-hek293t-pe2-replace.png}
        \label{fig:shap_replacement}
    }%
    \subfigure[Insertion]{
        \includegraphics[width=0.33\textwidth]{shap_1bp-dp-hek293t-pe2-insert.png}
        \label{fig:shap_insertion}
    }%
    \subfigure[Deletion]{
        \includegraphics[width=0.33\textwidth]{shap_1bp-dp-hek293t-pe2-delete.png}
        \label{fig:shap_deletion}
    }
    \caption[SHAP analysis for 1-bp insertion and deletion of the DeepPrime model trained on the HEK293T-PE2 dataset]{SHAP analysis for 1-bp insertion and deletion of the DeepPrime model trained on the HEK293T-PE2 dataset. The x-axis represents the attributed weight to the outcome (SHAP value), while the y-axis represents the features ranked by importance (mean absolute SHAP value). The colour of the points indicates the feature value, with red indicating high values and blue indicating low values.}
    \label{fig:shap-1bp}
\end{figure}

The attention weights for the edits of length 3 showed similar result (\autoref{fig:attention_deletion_3bp}), the only noticeable difference is that the high attention regions for deletion were extended to 3bp long, reflecting the longer deletion length.

During SHAP analysis in \autoref{sec:determinants}, the edited base was not investigated, as a meaningful representation for bases of different lengths could not be found. Thus, to understand if the base to edit during deletion is indeed an important feature, SHAP analysis was conducted again on the HEK293T PE2 dataset for 1-bp replacement, insertion, and deletion (\autoref{fig:shap-1bp})


Interestingly, different from the attention weights, the base to edit during deletion was not considered an important feature by the model (Figure \ref{fig:shap_deletion}). The same analysis was also conducted on the four PRIDICT datasets (\autoref{appendix:shap-1bp-pridict}), and the results were consistent. 
The same was true for the a562 pe2max dataset, where the transformer model significantly outperformed PRIDICT and DeepPrime. This to some extent undermines the improvement in performance, suggesting that the better result could be a result of randomness in the training process, rather than the inherent superiority of the model architecture.
\chapter{Discussion}

The cross attention mechanism is possibly more suitable for the case of base editing efficiency prediction, since the editing window is relatively static with regard to the protospacer location. While for prime editors, even by following the more stringent constraint on LHA length than practical standard, the edit position can vary in a window as wide as 20bp. This resulted in a far bigger possible range of possible outcomes

A number of alternative structures could be experimented with 


\startappendices
% Add or remove any appendices you'd like here:


%%%%% REFERENCES


{\renewcommand*\MakeUppercase[1]{#1}%

% \chapter{Summary and Comparison of Existing Learning Based Models}

\label{appendix:models}

\section{DeepPE}

Developed by Kim et al, DeepPE is one of the earliest attempt at predicting the outcomes of prime editing using machine learning and illustrated many possible determinants of PE efficiency. Most of the determinants discovered in their study are still valid today, but the model itself is not as relevant in terms of performance due to the constraints in datasets at the time of their publication. The model is also very limited in terms of editing types supported, focusing on predicting the efficiency of G to C substitution at the position +5 nick site of the target sequence. 

After the users provided the target sequence to edit, the web tool will propose a number of pegRNA sequences and evaluate their efficiency using the DeepPE model. The 47-nt long target sequence and the 17 to 37nt RTT+PBS sequences, as well as 20 explicit features including the GC content and melting temperature of the PBS are used as input to the model. The two nucleotide sequences are one-hot encoded into four dimensional matrices. The two embedded sequences and the explicit features are then concatenated(stacked) together and fed into a convolutional neural network with 10 $3 \times 4$ filters. The output is pooled using a deep reinforcement learning model instead of a traditional pooling layer, and then input into a fully connected layer with 1000 units. The result is linearly transformed to the DeepPE prediction score, indicating the efficiency of the editing process on the target sequence using the provided PBS and RTT.

A sketch of the model architecture is shown in \autoref{fig:deeppe}.

\begin{figure}[ht]
    \centering
    \includegraphics[width=0.6\textwidth]{DeepPE.png}
    \caption{DeepPE Model Architecture}
    \label{fig:deeppe}
\end{figure}

This architecture had the highest performance among the methods reviewed by the authors, although not significantly higher than L1 Lasso regression model. The model achieved a Spearman's R of 0.7 to 0.8, and a Pearson's r of 0.6 to 0.7 on the held-out as well as generalization(unseen) datasets.

The authors also developed another multi-layer perceptron(MLP) model to predict the editing efficiency of more general editing types, including single-nucleotide insertions, deletions and substitutions. However, random forest achieved better performances and was thus used in the additional PE\_Type and PE\_Position models. PE\_Type model proposes pegRNA for 24 possible edits, including specific single-nucleotide deletions, insertions, and substitutions at designated locations. PE\_Position model proposes pegRNA optimized to perform substitutions at ten more positions, namely positions 1, 2, 3, 4, 6, 7, 8, 9, 11 and 14.

\section{Easy Prime}

EasyPrime is a XGBoost regression model developed by Liu et al to produce design for RTT, PBS and ngRNA. Instead of arbitrary mutations, Easy-Prime predicts the editing efficiency of the variants logged in the Genome-Wide Association Studies(GWAS) database. The GWAS variants are the single nucleotide polymorphisms(SNPs) that have been associated with particular traits or diseases.

Similar to DeepPE, for each variant, the web tool proposes a number of pegRNA design using the constrains on PBS, RTT and ngRNA length provided by the user. The proposed pegRNAs are then evaluated by the XGBoost models to find the optimal candidate. When producing the efficiency score, EasyPrime takes the extracted features from pegRNA and target sequences as input to the model instead of the sequences themselves. The extracted features include GC content of the PBS, PAM sequence disruption, as well as several target mutation features describing the mutations to insert.

Cas9 activity score produced by DeepSpCas9 is also used as a feature in the model. DeepSpCas9 is a convolutional neural network model that predicts the activity of the SpCas9 protein on a given target sequence and sgRNA pair. The model architecture is very similar to DeepPE, with one convolutional layer followed by three fully connected layers. The unique feature of the model is the use of filters of different sizes in the convolutional layer(3, 5, and 7nt), allowing the model to capture the dependencies between the base pairs at different distances\cite{kimSpCas9ActivityPrediction2019}.

Albeit limited in the type of edits supported, EasyPrime is one of a few methods that provides official support for ngRNA design required by PE3 and PE3b. Note that PE3b is the optimized version of PE3, where the ngRNA is selected to avoid possible DSB by not targeting the nucleotide complementary to the nicked position\cite{liudavidr.SearchandreplaceGenomeEditing2019}.

Constrained by the data available and the simple architecture of the model, the performance of EasyPrime is relatively low compared to the other models reviewed. The model achieved a Spearman's R and Pearson's r of 0.5 to 0.6 on their held-out datasets.

\section{DeepPrime}

\begin{figure}[ht]
    \centering
    \includegraphics[width=0.8\textwidth]{DeepPrime.png}
    \caption{DeepPrime Model Architecture}
    \label{fig:deepprime}
\end{figure}


Also developed by Kim et al., DeepPrime is the updated version of DeepPE with an upgraded model architecture. 

Illustrated in \autoref{fig:deepprime}, instead of PBS + RTT with target sequence, it takes the wild type(unedited) as well as the edited sequences as input to the CNN. The convolutional network is significantly larger than DeepPE, containing four convolutional layers with 128, 108, 108, and 128 filters, respectively. Moreover, conventional average pooling is used this time round after each convolutional layer instead of a deep RL algorithm. Batch normalization is also applied to accelerate training.  The output of the convolutional layer is then processed by bidirectional Gated Recurrent Units(GRU). 

Instead of processed together with the embedded sequence, features extracted from the proposed pegRNA and target context sequence are processed using a separate four-layer feed forward neural network. The outputs from the two networks are then stacked and fed into a fully connected layer. The result is linearly transformed and processed by a SoftPlus activation function to produce the final prediction score.


DeepPrime is effectively a multi-task learning model, with a base model trained using the combination of 18 datasets of different PE and cell line combinations. The base model is then fine tuned using each of the 18 datasets to produce a task specific model for each setting, collectively named DeepPrime-FT.  

The amount of training data used makes DeepPrime the most comprehensive method in terms of the number of cell lines and PE versions covered. At the same time, the multi-task design allows DeepPrime-FT to utilize the share features between the different PEs and cell lines, and thus achieve very high performance. 

The model has made significant improvement compared to DeepPE, with a Spearman's R of 0.8 to 0.9, and a Pearson's r of 0.7 to 0.9 on most of the testing datasets, including the generalization datasets unseen during training.

\section{PRIDICT}

\begin{figure}[ht]
    \centering
    \includegraphics[width=\textwidth]{pridict.png}
    \caption{PRIDICT Model Architecture}
    \label{fig:pridict}
\end{figure}


Developed by Gerald Schwank et al, PRIDICT utilizes a sophisticated attention-based bidirectional RNN model with a similar pegRNA recommendation pipeline to DeepPE and DeepPrime. 

The model overall is a three encoder one decoder architecture. Two of the encoders are attention-based bidirectional RNN models, learning the vector representation of the sequence data. The third encoder is a feed forward neural network taking explicit features derived from the proposed pegRNA, such as the length of the modifications to insert and melting temperature of the PBS, as inputs, similar to DeepPrime.

The target and mutation sequences are one-hot encoded, alongside three additional binary encoding indicating whether the nucleotide belongs to the protospacer, RTT or PBS. The four embeddings are stacked together into a vector of length 9 for each token in the target sequence and 7 for the mutated sequence(the protospacer embedding is omitted for mutated sequence) and fed into the model.

The bidirectional RNN model is used to capture the dependencies between the base pairs within the whole sequence, instead of only past information captured by unidirectional RNN models. Two separate attention query vectors then pools(compresses) sequence of token-level representations into one fixed length vector using the calculated attention weights. One query vector pools all tokens of the sequences, providing context. The other pools only RTT tokens, focusing on the part where the edits are made. 

The decoder is another feed forward neural network with residual connections and layer normalization, taking the pooled vectors from the encoders and calculating the probability distribution of possible outcomes of the edits when using the proposed guide. 

The model architecture is illustrated in \autoref{fig:pridict}.

Significantly higher performance was achieved by PRIDICT when compared to DeepPE (including PE\_Type and PE\_Position) and EasyPrime, with 2-3 fold increase in Spearman's R and Pearson's r on the generalization datasets curated by Gerald et al. PRIDICT also achieved comparable or better results on the datasets DeepPE and EasyPrime were originally trained on. In terms of current generation of models, it was on par with DeepPrime on the HEK293T datasets when predicting intended edits, with a Spearman's R of 0.81. At the same time, PRIDICT outperformed the MinsePIE model as mentioned in section \ref{sec:minsepie}.

\chapter{Full List of Features Invested}
\label{appendix:features}

For individual feature such as tm-pbs, $\checkmark$ indicates that the feature was used in the final model and $\times$ indicates that the feature was not used in the final model. At the same time, for a group of features, if at least one feature was used in the final model, a $\circ$ is placed in the last column. 

% table of features and their explanations
\begin{table}[ht]
    \centering
    \begin{tabular}{|p{0.3\textwidth}|p{0.5\textwidth}|p{0.1\textwidth}|}
        \hline
        \textbf{Feature} & \textbf{Explanation} & \textbf{Top 24} \\ 
        \hline
        tm-pbs& Melting temperature fo pbs RNA sequences & $\checkmark$ \\
        \hline
        tm-rt& Melting temperature of RT primer RNA sequences & $\times$ \\
        \hline
        tm-spacer& Melting temperature of spacer RNA sequences & $\checkmark$ \\
        \hline
        max-cas & Maximum number of consecutive A bases in the cDNA of extension as well as protospacer sequence & $\times$ \\
        \hline
        max-cts & Maximum number of consecutive T bases in the cDNA of extension as well as protospacer sequence & $\checkmark$ \\
        \hline
        max-cgs & Maximum number of consecutive G bases in the cDNA of extension as well as protospacer sequence & $\times$ \\
        \hline
        max-ccs & Maximum number of consecutive C bases in the cDNA of extension as well as protospacer sequence & $\times$ \\
        \hline
    \end{tabular}
\end{table}

% table of features and their explanations
\begin{table}[ht]
    \centering
    \begin{tabular}{|p{0.3\textwidth}|p{0.5\textwidth}|p{0.1\textwidth}|}
        \hline
        \textbf{Feature} & \textbf{Explanation} & \textbf{Top 24} \\ 
        \hline
        mfe-pbs & Minimum free energy of pbs RNA sequences & $\checkmark$ \\
        \hline
        mfe-rt & Minimum free energy of RT primer RNA sequences & $\times$ \\
        \hline
        mfe-spacer & Minimum free energy of spacer RNA sequences & $\checkmark$ \\
        \hline
    \end{tabular}
\end{table}

\chapter{Fine Tuning the Transformer Model}
\label{appendix:transformer-hyperparameter-tuning}

Full result of the hyperparameter tuning for the transformer model.

\begin{longtable}{|>{\raggedright\arraybackslash}p{2cm}|>{\raggedright\arraybackslash}p{2cm}|>{\raggedright\arraybackslash}p{2cm}|>{\raggedright\arraybackslash}p{3.5cm}|>{\raggedright\arraybackslash}p{2cm}|>{\raggedright\arraybackslash}p{2cm}|}
    \hline
    \textbf{MLP Embed Dim} & \textbf{Num Encoder Units} & \textbf{Dropout} & \textbf{Performance} & \textbf{Mean} & \textbf{Max} \\
    \hline
    \endfirsthead
    
    \hline
    \textbf{MLP Embed Dim} & \textbf{Num Encoder Units} & \textbf{P Dropout} & \textbf{Performance} & \textbf{Mean Performance} & \textbf{Max Performance} \\
    \hline
    \endhead
    
    \hline
    \endfoot
    
    100 & 1 & 0.1 & [0.8039, 0.8214, 0.8204, 0.7896, 0.7923] & 0.8055 & 0.8214 \\
    \hline
    100 & 1 & 0.3 & [0.8062, 0.8024, 0.8111, 0.8009, 0.8258] & 0.8093 & 0.8258 \\
    \hline
    100 & 1 & 0.5 & [0.7753, 0.7982, 0.7899, 0.8025, 0.7869] & 0.7906 & 0.8025 \\
    \hline
    100 & 2 & 0.1 & [0.8035, 0.8024, 0.7926, 0.8187, 0.7899] & 0.8014 & 0.8187 \\
    \hline
    100 & 2 & 0.3 & [0.8137, 0.8168, 0.8014, 0.7962, 0.7909] & 0.8038 & 0.8168 \\
    \hline
    100 & 2 & 0.5 & [0.7986, 0.8008, 0.7874, 0.7776, 0.7771] & 0.7883 & 0.8008 \\
    \hline
    100 & 3 & 0.1 & [0.7995, 0.8001, 0.8015, 0.8003, 0.8128] & 0.8028 & 0.8128 \\
    \hline
    100 & 3 & 0.3 & [0.7879, 0.8043, 0.7797, 0.7929, 0.7939] & 0.7917 & 0.8043 \\
    \hline
    100 & 3 & 0.5 & [0.7735, 0.7864, 0.7758, 0.7888, 0.7847] & 0.7818 & 0.7888 \\
    \hline
    150 & 1 & 0.1 & [0.8047, 0.8202, 0.8059, 0.8001, 0.7957] & 0.8053 & 0.8202 \\
    \hline
    150 & 1 & 0.3 & [0.8195, 0.8171, 0.8085, 0.7845, 0.7858] & 0.8031 & 0.8195 \\
    \hline
    150 & 1 & 0.5 & [0.7554, 0.7521, 0.7853, 0.7884, 0.7979] & 0.7758 & 0.7979 \\
    \hline
    150 & 2 & 0.1 & [0.7995, 0.8025, 0.7994, 0.7979, 0.7990] & 0.7996 & 0.8025 \\
    \hline
    150 & 2 & 0.3 & [0.7998, 0.7934, 0.7941, 0.8122, 0.8094] & 0.8018 & 0.8122 \\
    \hline
    150 & 2 & 0.5 & [0.7938, 0.7935, 0.7799, 0.7985, 0.7643] & 0.7860 & 0.7985 \\
    \hline
    150 & 3 & 0.1 & [0.8033, 0.7837, 0.8062, 0.8033, 0.8046] & 0.8002 & 0.8062 \\
    \hline
    150 & 3 & 0.3 & [0.7946, 0.7983, 0.7940, 0.8186, 0.7984] & 0.8008 & 0.8186 \\
    \hline
    150 & 3 & 0.5 & [0.7756, 0.7873, 0.7961, 0.7878, 0.7923] & 0.7878 & 0.7961 \\
    \hline
    200 & 1 & 0.1 & [0.8007, 0.8062, 0.8216, 0.8095, 0.8182] & 0.8112 & 0.8216 \\
    \hline
    200 & 1 & 0.3 & [0.7820, 0.8164, 0.7751, 0.8053, 0.8172] & 0.7992 & 0.8172 \\
    \hline
    200 & 1 & 0.5 & [0.7951, 0.7843, 0.7835, 0.7956, 0.8046] & 0.7926 & 0.8046 \\
    \hline
    200 & 2 & 0.1 & [0.8058, 0.7912, 0.7984, 0.8011, 0.7862] & 0.7965 & 0.8058 \\
    \hline
    200 & 2 & 0.3 & [0.7999, 0.7955, 0.7863, 0.8036, 0.7992] & 0.7969 & 0.8036 \\
    \hline
    200 & 2 & 0.5 & [0.7867, 0.7715, 0.7559, 0.7856, 0.7680] & 0.7735 & 0.7867 \\
    \hline
    200 & 3 & 0.1 & [0.8040, 0.8018, 0.8117, 0.8252, 0.8054] & 0.8096 & 0.8252 \\
    \hline
    200 & 3 & 0.3 & [0.8034, 0.7934, 0.7644, 0.8073, 0.7846] & 0.7906 & 0.8073 \\
    \hline
    200 & 3 & 0.5 & [0.7701, 0.7940, 0.7706, 0.7858, 0.7865] & 0.7814 & 0.7940 \\
    \hline
    \caption{Transformer model's performance using different hyperparameters controlling the parameter size. The hyperparameters are the MLP Embedding Dimension (number of hidden units in the positional feedforward network), Number of Encoder Units (number of transformer layers to stack), and the Dropout Rate. The raw performance as well as the mean and maximum performance are shown.} 
    \end{longtable}
    


\chapter{Additional Figures}
\label{appendix:additional-figures}

\section{Additional Figures for Error Analysis}
\label{appendix:error-analysis-figures}

\section{}


\printbibliography[heading=bibintoc,title={\bibtitle}]}


\end{document}